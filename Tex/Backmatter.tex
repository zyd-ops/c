% !TEX root = ../Thesis.tex

%---------------------------------------------------------------------------%
%->> Backmatter
%---------------------------------------------------------------------------%
\chapter[致谢]{致\quad 谢}\chaptermark{致\quad 谢}% syntax: \chapter[目录]{标题}\chaptermark{页眉}
%\thispagestyle{noheaderstyle}% 如果需要移除当前页的页眉
%\pagestyle{noheaderstyle}% 如果需要移除整章的页眉

首先真心感谢国科大杭高院。在工作了好几年之后,能有机会从职场的快节奏里抽身,重新回到学校安安静静地读几年书,这份际遇对我来说太珍贵了。

很荣幸能成为杜阳老师回国执教后的首届学生,十分感谢杜老师在硕士期间给予的指导和帮助。您高超的学术水平、充满干劲的科研态度以及非常有规律的作息,是我永远的学习榜样。

三年,说长不长,说短也不短。回想当初,复试表现欠佳,内心忐忑不安,是王宁老师的信任与包容,让我侥幸通过拟录取,得以延续这段求学缘分。这份知遇之恩,才让我有了这三年宝贵的校园经历。

很庆幸遇到唯一的同门叶志诚(小叶)。从第一次见面那个自带喜感的自我介绍开始,你就是我这三年的快乐源泉。平时虽嘻嘻哈哈,但关键时刻你是真能扛事。

临近毕业,依然常觉不自信。但站在即将三十一岁的路口,对“学历”这个光环渐渐祛魅了——它不是万能的,但这段经历确实让我变得更好了。

未来路长,希望能保持这份平和,继续前行。

\vspace{3\baselineskip}

\rightline{2026年1月}
\chapter{作者简历及攻读学位期间发表的学术论文与其他相关学术成果}

\section*{作者简历:}
2013年09月——2017年06月,在武汉科技大学计算机学院获得学士学位。

2023年09月——2026年06月,在中国科学院大学杭州高等研究院物理与光电工程学院获得硕士学位。

\section*{已发表(或正式接受)的学术论文:}
{
\setlist[enumerate]{}% restore default behavior
\begin{enumerate}[nosep]
    \item \textbf{Xu J}, Ye Z, Wang S, et al. Dual holographic and polarization encoding for high fidelity image transmission through multimode fibers[J]. Optics \& Laser Technology, 2025, 191: 113301.
    \item Ye Z, \textbf{Xu J}, Wang S, et al. Computational immunity-belt reconstruction restores far-field images through bent step-index fibers[J]. Photonics Research, 2025, 13(11): 3199-3209.
    \item Huang S, Ye Z, \textbf{Xu J}, et al. Force Sensing in Multimode Fibers via Wavefront-Shaped Focal-Spot Analysis[J]. IEEE Sensors Journal, 2025.
\end{enumerate}
}

\section*{申请或已获得的专利:}
{
\setlist[enumerate]{}% restore default behavior
\begin{enumerate}[nosep]
    \item 基于多模光纤全息和偏振编码的图像重建方法及成像系统,第二作者,申请号:202510705774.2
\end{enumerate}
}

\cleardoublepage[plain]% 让文档总是结束于偶数页,可根据需要设定页眉页脚样式,如 [noheaderstyle]
%---------------------------------------------------------------------------%
