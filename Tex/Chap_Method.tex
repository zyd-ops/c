% !TEX root = ../Thesis.tex

\chapter{基于双编码的多模光纤成像实验研究}\label{chap:method}

{
本章详细介绍本文提出的基于全息编码和偏振编码的双编码多模光纤成像方法,这是本论文的核心创新内容。第\ref{chap:preliminary}章基于公开数据集验证了深度学习方法在散斑重建中的有效性,本章将进一步提出双编码方案以提升多模光纤的信息传输容量,并通过自建实验系统进行验证。

\section{引言}\label{sec:method_intro}

多模光纤(MMF)因其能够支持数千个空间模式,在紧凑且微创的形态下实现高分辨率成像,成为生物医学内窥成像和光学通信领域的理想探针。然而,光在多模光纤中传输时,模间耦合、模式干涉以及侧壁反射会导致角谱畸变,在光纤远端产生复杂的散斑图案,严重限制了图像传输的保真度。

第\ref{chap:preliminary}章表明,基于深度学习的端到端重建方法能够有效建立散斑与目标图像间的映射关系。然而,如何进一步提升多模光纤的信息容量和重建保真度,充分利用光纤支持的数千个模式,仍是亟待解决的关键问题。

为此,本章提出一种\textbf{双编码(Dual Encoding)}方案:同时利用\textbf{全息编码(Holographic Encoding)}和\textbf{偏振编码(Polarization Encoding)}来增强多模光纤的信息承载能力。两种编码信息均通过计算全息图的生成算法预先设计和确定,无需在数据采集过程中调整硬件光路。具体而言:
\begin{itemize}
    \item \textbf{全息编码}:在计算全息图生成时,通过设计不同的空间频率和相位分布,将目标图像编码到傅里叶平面的不同空间位置(全息标签),使得DMD产生的+1级衍射光以不同的入射角度耦合进光纤,从而激发不同的模式组合。
    \item \textbf{偏振编码}:通过精确控制计算全息图的衍射光束在傅里叶平面的空间坐标,利用固定配置的偏振分束器(Beam Displacer, BD)的空间分离特性,使衍射光选择性地通过BD的不同偏振通道(s偏振、p偏振或混合偏振),为每幅图像赋予特定的偏振标签。
\end{itemize}

这种双编码策略使得散斑图案同时包含空间信息和偏振信息,显著增强了散斑模式的可区分性和信息容量。

本章的主要创新点和贡献包括:
\begin{enumerate}
    \item \textbf{双模态编码方案}:首次将全息编码与偏振编码相结合,显著提升多模光纤的信息传输容量,理论上可实现$25 \times 3 = 75$路复用。
    
    \item \textbf{改进的神经网络架构}:提出DeepLeakyU-Net (DLU-Net)网络,通过在编码器中采用Leaky ReLU激活函数、使用步长卷积替代池化层、增加网络深度和基础特征数,改善深层编码器的梯度流动和空间信息保留,提升重建性能。
    
    \item \textbf{偏振鲁棒性验证}:在s偏振、p偏振和混合偏振三种条件下均实现了高保真度重建(平均SSIM = 0.93),证明方法的偏振不变性和鲁棒性。
    
    \item \textbf{无监督聚类分析}:利用UMAP和t-SNE等无监督降维方法,验证不同偏振态和全息标签下的散斑图在特征空间中形成清晰可分的聚类簇,为多路复用和分类重建奠定了物理基础。
    
    \item \textbf{远场成像机制}:不同于传统的近场散斑成像,本文在光纤远场采集散斑图案,利用阶跃型多模光纤的角度相关性,实现了基于角域信息的图像重建。
\end{enumerate}

本章结构安排如下:第\ref{sec:dual_encoding}节介绍双编码方案的原理与实现;第\ref{sec:optical_setup}节描述实验光学系统的搭建;第\ref{sec:dataset_construction}节详细说明数据集的构建与特征;第\ref{sec:dlu_net}节提出改进的DeepLeakyU-Net网络架构;第\ref{sec:training_details}节介绍训练策略与实现细节;第\ref{sec:clustering_analysis}节进行散斑模式的无监督聚类分析;第\ref{sec:experimental_results}节展示DLU-Net重建结果与定量评估;第\ref{sec:discussion_method}节讨论方法的优势与局限;第\ref{sec:method_summary}节对本章进行总结。


\section{双编码方案原理与实现}\label{sec:dual_encoding}

\subsection{编码方案概述}\label{sec:encoding_overview}

传统的多模光纤成像方法通常仅利用单一维度的信息编码(如空间位置或强度分布),限制了光纤的信息传输容量。为充分利用多模光纤支持的数千个空间模式,本文提出\textbf{双编码(Dual Encoding)}方案:同时利用\textbf{全息编码}和\textbf{偏振编码},将目标图像信息嵌入到空间维度和偏振维度。

双编码方案的核心思想是通过计算全息图的精确设计,在光学系统固定配置下实现可编程的双维度编码:
\begin{enumerate}
    \item \textbf{全息编码}:利用数字微镜器件(DMD)加载计算生成的二值振幅全息图,采用离轴全息配置精确控制+1级衍射光的波前和空间位置。通过在计算全息图生成算法中设计$5\times5$网格排列的全息标签(共25个),每个标签对应傅里叶平面上不同的空间坐标,从而在光纤近端产生不同的入射角度,激发不同的模式组合。
    
    \item \textbf{偏振编码}:通过计算全息图设计控制+1级衍射光在傅里叶平面的精确位置,利用固定配置的偏振分束器(Beam Displacer, BD)的空间分离特性(3.5 mm位移),使衍射光选择性地通过BD的不同偏振通道。四分之一波片(QWP)将两个正交线偏振态转换为左旋和右旋圆偏振态,实现偏振复用。本文考虑三种偏振态:s偏振、p偏振和混合偏振(s+p),均通过计算全息图设计实现。
\end{enumerate}

这种双编码策略的优势在于:\textbf{所有编码信息(全息标签和偏振标签)在计算全息图生成时预先确定,光学系统保持固定配置,无需在数据采集过程中调整硬件}。这使得散斑图案同时包含空间信息和偏振信息,显著增强了散斑模式的可区分性和信息容量,同时保证了系统的稳定性和数据采集的高效性。理论上,25个全息标签与3种偏振态的组合可实现$25 \times 3 = 75$路复用传输。

\subsection{全息编码原理}\label{sec:holographic_encoding}

全息编码的目的是将目标图像调制到特定的空间位置,从而在多模光纤输入端产生不同的入射角度和模式激发分布。与传统方法直接将图像投影到空间光调制器不同,本文采用\textbf{离轴全息配置(Off-axis Holographic Configuration)},利用DMD生成二值振幅光栅,通过精确控制衍射级次实现波前调制。

\subsubsection{DMD离轴全息原理}

数字微镜器件(DMD)由数百万个可独立控制的微镜组成,每个微镜可在$\pm 12^\circ$两个角度状态之间快速切换。通过设计特定的二值振幅图案,DMD可以产生多个衍射级次。本文仅选择+1级衍射光耦合进多模光纤,以实现高效的波前调制。

根据文献报道,+1级衍射光的空间坐标需要精心选择以优化衍射效率。本实验中,DMD上的衍射级次坐标被设置为$(73, 334)$和$(326, 79)$,DMD的有效显示区域定义为$624\times624$像素,所有全息图案均在此区域内显示。

\begin{figure}[!htbp]
    \centering
    \includegraphics[width=0.6\textwidth]{Img/4/DMD1.pdf}
    \bicaption{\enspace DMD上衍射级次坐标示意图。}{\enspace Schematic diagram of diffraction order coordinates on DMD.}
    \label{fig:dmd_coordinates}
\end{figure}

图\ref{fig:dmd_coordinates}展示了DMD上的坐标系统和衍射级次的空间分布。+1级和-1级衍射光分别位于对称位置,本实验选择+1级衍射光进行后续的偏振编码和光纤耦合。

\subsubsection{全息标签设计}

为增加散斑模式的多样性和实现空间复用,本文设计了\textbf{25个全息标签},排列为$5\times5$网格结构,每个标签之间间隔5个像素。这些全息标签对应傅里叶平面上不同的空间坐标$(x_i, y_j)$,其中$i, j \in \{1, 2, 3, 4, 5\}$。DMD上的标签坐标偏移集合定义为:

\[
S_{offset}^{DMD}=\left\{
\begin{aligned}
& (-10,10),\ (-5,10),\ (0,10),\ (5,10),\ (10,10), \\
& (-10,5),\ (-5,5),\ (0,5),\ (5,5),\ (10,5), \\
& (-10,0),\ (-5,0),\ (0,0),\ (5,0),\ (10,0), \\
& (-10,-5),\ (-5,-5),\ (0,-5),\ (5,-5),\ (10,-5), \\
& (-10,-10),\ (-5,-10),\ (0,-10),\ (5,-10),\ (10,-10)
\end{aligned}
\right\}
\]

\begin{itemize}
  \item \( pol1: \)  (73, 334) + \(S_{offset}^{DMD}\)
  \item \( pol2: \)  (326, 79) + \(S_{offset}^{DMD}\)
  \item \( pol3: \)  \( pol1 \) + \( pol2 \)
\end{itemize}

在生成计算全息图时,通过设计不同的空间频率分布,可以控制DMD产生的+1级衍射光在傅里叶平面的落点位置。不同的全息标签对应不同的衍射角度,使得目标图像以不同的入射角度耦合进光纤近端,激发不同的模式组合。由于阶跃型多模光纤的角度相关性,这些不同的入射角度会在光纤远场产生具有不同统计特性的散斑图案,从而实现基于角域信息的图像编码和重建。

\subsection{偏振编码原理}\label{sec:polarization_encoding}

偏振编码利用光的偏振态作为额外的信息维度。多模光纤支持正交的两个偏振模式,通过控制入射光的偏振态,可以为图像信息赋予偏振标签,进一步增强散斑图案的多样性。

\subsubsection{偏振态定义}

本文考虑三种偏振态:
\begin{itemize}
    \item \textbf{s偏振(s-polarization)}:电场矢量垂直于入射面振动,对应于水平线偏振态。
    \item \textbf{p偏振(p-polarization)}:电场矢量平行于入射面振动,对应于垂直线偏振态。
    \item \textbf{混合偏振(s+p)}:s偏振和p偏振分量同时存在,通过偏振分束器(BD)将两个正交偏振分量合并到同一光路中。
\end{itemize}

偏振态的差异会导致光纤中不同偏振模式的激发和耦合,从而产生具有不同统计特性的散斑图案。研究表明,圆偏振态在阶跃型多模光纤中具有较好的保持性,因此本文在光纤耦合前使用四分之一波片(QWP)将线偏振态转换为圆偏振态。

\subsubsection{偏振编码实现}

偏振编码通过固定配置的光学元件组合和计算全息图的精确设计共同实现:

\textbf{固定光学系统配置:}
\begin{enumerate}
    \item \textbf{半波片(HWP1)}:放置在DMD之前,确保入射线偏振与微镜的最佳方向对齐,最大化衍射效率并最小化DMD引入的椭圆偏振畸变。
    
    \item \textbf{线偏振片(LP1)}:滤除衍射过程中产生的不需要的正交偏振分量,确保偏振纯度。
    
    \item \textbf{半波片(HWP2)}:将透射光的偏振方向调整为相对于偏振分束器(BD)主轴$45^\circ$,使得入射光包含两个相等的正交偏振分量。
    
    \item \textbf{偏振分束器(BD)}:利用双折射晶体的特性,根据入射光束的空间位置和偏振态,将光束分离或合并。BD产生3.5 mm的空间位移,将两个正交偏振分量分离到傅里叶平面的不同区域。
    
    \item \textbf{四分之一波片(QWP1)}:在光纤耦合前,将两个正交线偏振态分别转换为左旋和右旋圆偏振态,以提高偏振态在光纤传输过程中的稳定性。
\end{enumerate}

\textbf{计算全息图设计控制:}

在生成计算全息图时,通过精确设计二值振幅光栅的空间频率和相位分布,可以控制+1级衍射光在傅里叶平面的空间坐标。利用BD的空间分离特性(3.5 mm位移),不同的全息图设计可以使衍射光束:
\begin{itemize}
    \item \textbf{仅通过s偏振通道}:衍射光落在傅里叶平面的特定区域A,仅s偏振分量被选择性耦合
    \item \textbf{仅通过p偏振通道}:衍射光落在空间位移3.5 mm的区域B,仅p偏振分量被选择性耦合
    \item \textbf{同时通过s+p偏振通道}:衍射光设计为在两个区域同时产生,s和p偏振分量同时耦合并合并
\end{itemize}

因此,目标图像的偏振标签(s偏振、p偏振或混合偏振)完全由计算全息图的设计决定,数据采集过程中光学系统保持固定配置,无需物理调整任何光学元件。

\subsection{双编码的信息容量分析}\label{sec:capacity_analysis}

双编码方案显著提升了多模光纤的信息传输容量。假设全息编码提供$N_{\text{holo}}$个空间标签,偏振编码提供$N_{\text{pol}}$个偏振态,则理论上可实现$N_{\text{holo}} \times N_{\text{pol}}$路复用传输。

在本文的实验中,全息标签数量$N_{\text{holo}} = 25$($5\times5$网格排列),偏振态数量$N_{\text{pol}} = 3$(s偏振、p偏振、混合偏振),因此理论上可实现$25 \times 3 = 75$路复用传输。图\ref{fig:dual_encoding_principle}展示了双编码方案的原理示意图。

\begin{figure}[!htbp]
    \centering
    \includegraphics[width=0.90\textwidth]{Img/4/dual-encoding-principle.png}
    \bicaption{\enspace 双编码方案原理示意图。展示了全息编码和偏振编码如何同时作用于目标图像(以数字'5'为例),产生不同的散斑图案,并通过深度神经网络(DNN)重建原始图像。图中展示了s偏振、p偏振和混合偏振三种偏振态下的编码和重建过程。}{\enspace Schematic illustration of the dual-encoding scheme. Shows how holographic and polarization encoding simultaneously act on target images (digit '5' as example), generating different speckle patterns that are reconstructed via deep neural network (DNN). Three polarization states (s-pol, p-pol, and mixed) are demonstrated.}
    \label{fig:dual_encoding_principle}
\end{figure}

\subsubsection{光纤模式容量}

本实验使用的阶跃型多模光纤参数为:纤芯直径$d = 100~\mu\mathrm{m}$,数值孔径$\text{NA} = 0.22$,工作波长$\lambda = 532~\mathrm{nm}$。根据多模光纤支持的模式数量估算公式:
\begin{equation}
M \approx \frac{4}{\pi} \left(\frac{\pi d \cdot \text{NA}}{\lambda}\right)^2,
\end{equation}
可计算得到该光纤支持约$M \approx 8439$个空间模式,远大于MNIST数据集单张图像的像素数($64\times64 = 4096$像素),因此光纤的模式容量足以支持本文的图像传输需求。

\subsubsection{远场角度相关性}

阶跃型多模光纤的一个重要特性是其远场分布与近场入射角度之间存在稳定的相关性。光功率在近端以特定角度区域入射,会在远端的对应角度区域出现,模间混合较小。这种角度相关性为本文的全息编码提供了物理基础:不同全息标签对应的不同入射角度会在远场产生可区分的散斑图案。

本文采用远场成像方式采集散斑图案,不同于传统方法直接对光纤近场成像。远场散斑图案保留了编码的角度信息,通过模式干涉隐式嵌入在散斑图案中,使得后续的深度学习网络能够有效解码和重建原始图像。

双编码产生的散斑图案具有更高的多样性和可区分性,为后续的深度学习重建和无监督聚类分析提供了丰富的特征空间。


\section{实验光学系统搭建}\label{sec:optical_setup}

\subsection{光学系统设计}\label{sec:optical_design}

本节详细描述实验光学系统的设计与搭建。图\ref{fig:optical_setup}为实验光路示意图,展示了从光源到探测器的完整光路,包括全息编码模块、偏振编码模块、光纤耦合系统和远场成像系统。

\begin{figure}[!htbp]
    \centering
    \includegraphics[width=0.95\textwidth]{Img/4/optical-setup.png}
    \bicaption{\enspace 实验光学系统示意图。光路包括:光源系统(532 nm激光器+保偏光纤PMF)、DMD全息编码模块、偏振控制与编码模块(HWP1、LP1、HWP2、BD、QWP1)、光纤耦合系统(L2、L3、L4、Obj1)和远场成像系统(Obj2、L5、L6、相机)。MMF为阶跃型多模光纤(纤芯100 μm,NA=0.22,长度20 cm)。}{\enspace Schematic diagram of the experimental optical setup, including light source (532 nm laser + PMF), DMD holographic encoding, polarization control modules (HWP1, LP1, HWP2, BD, QWP1), fiber coupling system (L2, L3, L4, Obj1), and far-field imaging system (Obj2, L5, L6, Camera). MMF: step-index multimode fiber (core 100 μm, NA=0.22, length 20 cm).}
    \label{fig:optical_setup}
\end{figure}

实验光学系统主要由以下几个部分组成:

\subsubsection{光源系统}

光源采用线偏振二极管激光器(波长$\lambda = 532~\mathrm{nm}$,功率200 mW,型号:MSL-U-532,CNI公司)。激光器输出光束首先耦合进保偏光纤(PMF),以提高光束质量和偏振稳定性。保偏光纤输出的光束以$24^\circ$入射角照明DMD表面。

在DMD之前放置半波片HWP1,用于调整入射线偏振方向,使其与DMD微镜的最佳方向对齐,从而最大化衍射效率并最小化DMD引入的椭圆偏振畸变。透镜L1(焦距$f = 100~\mathrm{mm}$,型号:OLD2444-T2M,JCOPTIX)用于准直和扩束,使光束均匀覆盖DMD表面。

\subsubsection{全息编码模块:数字微镜器件(DMD)}

\textbf{数字微镜器件(Digital Micromirror Device, DMD)}是实现全息编码的关键元件。DMD由数百万个可独立控制的微镜组成,每个微镜可在$\pm 12^\circ$两个角度状态之间快速切换,从而实现对入射光的振幅调制。

在本实验中,DMD用于加载计算生成的二值振幅全息图,实现对目标图像的空间位置编码。DMD的有效显示区域设置为$624\times624$像素,所有全息图案均在此区域内显示。通过精确设计二值振幅光栅,DMD产生的+1级衍射光被选择性地耦合进多模光纤。

\subsubsection{偏振编码模块}

偏振编码模块由一系列偏振控制元件组成,实现对入射光偏振态的精确调控:

\begin{enumerate}
    \item \textbf{线偏振片LP1}(型号:FLP25-VIS-M,LBTEK):滤除DMD衍射过程中产生的不需要的正交偏振分量,确保偏振纯度。
    
    \item \textbf{半波片HWP2}(型号:HWP25-532A-M,LBTEK):将透射光的偏振方向旋转$45^\circ$,使其相对于偏振分束器(BD)的主轴成$45^\circ$角,为后续的偏振分离做准备。
    
    \item \textbf{偏振分束器BD}(型号:BD110-35-SP,LBTEK):利用双折射晶体的特性,将入射光束按偏振态分离为两束空间位移3.5 mm的正交线偏振光,或将两束正交偏振光合并为一束。DMD上的精确二值振幅光栅设计使得在傅里叶平面产生恰好3.5 mm的光束位移,与BD的空间分离相匹配,从而实现两束光的重新合并并保持正交偏振态。
    
    \item \textbf{四分之一波片QWP1}(型号:QWP25-532A-M,LBTEK):在光纤耦合前,将两个正交线偏振态分别转换为左旋和右旋圆偏振态。圆偏振态在阶跃型多模光纤中具有较好的保持性,有利于提高偏振编码的稳定性。
\end{enumerate}

\subsubsection{光纤耦合与成像系统}

DMD编码后的全息图案通过两个连续的4$f$成像系统(由透镜L2、L3、L4和显微物镜Obj1组成)进行中继和缩小,最终耦合到多模光纤的近端面。具体光学元件参数如下:
\begin{itemize}
    \item L2:焦距$f = 200~\mathrm{mm}$(型号:OLD2466-T2M,JCOPTIX)
    \item L3:焦距$f = 80~\mathrm{mm}$(型号:OLD2441-T2M,JCOPTIX)
    \item L4:焦距$f = 150~\mathrm{mm}$(型号:OLD2454-T2M,JCOPTIX)
    \item Obj1:显微物镜,40倍,数值孔径0.65(Olympus)
\end{itemize}

\subsubsection{多模光纤}

本实验使用商用阶跃型多模光纤,主要参数如下:
\begin{itemize}
    \item 光纤类型:阶跃型多模光纤(Step-index MMF)
    \item 纤芯直径:$100~\mu\mathrm{m}$
    \item 包层直径:$125~\mu\mathrm{m}$
    \item 数值孔径:$\text{NA} = 0.22$
    \item 光纤长度:$20~\mathrm{cm}$
    \item 支持的模式数量:约8439个空间模式
\end{itemize}

阶跃型多模光纤的纤芯为均匀介质,支持稳定的传播不变模式。其远场强度分布呈现同心环状结构,环的直径随模式传播常数的减小而增大。这种特性建立了近场和远场分布之间稳定的角度相关性:在近端以特定角度区域入射的光功率,会在远端的对应角度区域出现,模间混合较小。

\subsubsection{远场探测与成像系统}

不同于传统方法直接对光纤近场成像,本文采用远场成像方式采集散斑图案。光纤远端的输出光场通过显微物镜Obj2(40倍,0.65 NA,Olympus)和两个消色差双透镜L5、L6成像到相机上:
\begin{itemize}
    \item L5:焦距$f = 100~\mathrm{mm}$(型号:OLD2444-T2M,JCOPTIX)
    \item L6:焦距$f = 40~\mathrm{mm}$(型号:OLD2434-T2M,JCOPTIX)
\end{itemize}

相机采集的远场散斑图案尺寸为$128\times128$像素。值得注意的是,携带全息和偏振编码信息的相干光束以不同角度入射到光纤近端面,在远场输出端产生独特的干涉图案。这些散斑图案通过模式干涉隐式保留了编码的角度信息,使得后续的深度学习网络能够有效解码和重建原始图像。

\subsection{光路对准与校准}\label{sec:alignment}

为保证实验的稳定性和可重复性,光路的对准与校准至关重要。主要步骤包括:
\begin{enumerate}
    \item \textbf{光源准直}:调整保偏光纤输出端位置,确保光束平行且均匀入射到DMD表面,入射角度为$24^\circ$。
    
    \item \textbf{DMD调制验证}:在DMD上加载测试全息图,使用光功率计测量不同衍射级次的能量分布,验证+1级衍射光的位置和效率。
    
    \item \textbf{偏振态校准}:在BD后放置旋转偏振片,测量透射光强随偏振片角度的变化,验证两束正交偏振光的偏振纯度和消光比(要求消光比$>100:1$)。
    
    \item \textbf{光纤耦合优化}:使用六维调整架精确调整光纤近端面的位置和角度,监测光纤远端输出光功率,通过迭代优化使耦合效率最大化(耦合效率约60\%)。
    
    \item \textbf{远场成像系统校准}:调整Obj2和透镜L5、L6的位置,确保光纤远场分布清晰成像到相机上。通过观察远场同心环状结构验证成像质量。
    
    \item \textbf{散斑图采集优化}:调整相机曝光时间(典型值:10-50 ms)和增益,确保散斑图对比度适中(平均灰度值约为动态范围的40\%-60\%)且无像素饱和。
\end{enumerate}

\subsection{系统稳定性与环境控制}\label{sec:stability}

多模光纤对环境扰动(如温度、机械振动、弯曲等)极为敏感,散斑图案会随光纤形变而变化。为提高系统稳定性,本实验采取以下措施:
\begin{itemize}
    \item \textbf{隔振措施}:将整个光学系统安装在气浮隔振台上(隔振频率$<1~\mathrm{Hz}$),有效隔离地面振动和环境噪声。
    
    \item \textbf{光学暗室}:实验在全封闭光学暗室中进行,隔绝环境光干扰,确保相机仅采集光纤输出的散斑信号。
    
    \item \textbf{光纤固定}:光纤两端使用光纤夹具牢固固定在光学平台上,中间段保持自然悬空状态,避免外力接触。光纤保持笔直配置,未引入弯曲或扭转。
    
    \item \textbf{温度控制}:实验室配备恒温空调系统,温度控制在$23 \pm 1^\circ\mathrm{C}$,相对湿度控制在$40\%-60\%$,减小温度漂移对光纤传输特性的影响。
    
    \item \textbf{系统预热}:激光器和相机在数据采集前预热至少30分钟,确保光功率和探测器响应稳定。
    
    \item \textbf{数据采集策略}:考虑到通过LabVIEW程序将全息图上传至DMD过程中可能存在的系统稳定性问题和随机噪声影响,每一批数据采集时,前九张散斑图作为占位符舍弃,有效数据从第十张散斑图开始记录。这一策略有效避免了DMD加载全息图初期的不稳定状态和瞬态噪声对数据质量的影响。整个数据集的采集在连续的时间段内完成(约2-3天),避免长时间间隔导致的系统漂移。
\end{itemize}

需要特别指出的是,本实验中光纤保持固定的笔直配置,以确保散斑图案的稳定性和可重复性。在实际应用场景(如生物医学内窥成像)中,光纤会经历弯曲、温度变化和机械扰动,这些动态条件下的散斑图案会发生变化,可能需要实时重新校准或自适应网络重训练以保持重建保真度。这是未来研究需要解决的重要问题。


\section{数据集构建}\label{sec:dataset_construction}

\subsection{数据集设计策略}\label{sec:dataset_design}

为充分验证双编码方案的有效性,本文构建了一个包含多种全息标签和偏振态组合的大规模散斑-图像配对数据集。数据集的设计遵循以下原则:
\begin{enumerate}
    \item \textbf{多样性}:涵盖不同的目标图像、全息标签和偏振态组合,确保网络能够学习到丰富的散斑-图像映射关系。
    \item \textbf{平衡性}:各类别样本数量尽量均衡,避免类别不平衡导致的训练偏差。
    \item \textbf{可重复性}:记录每个样本的编码参数(全息坐标、偏振态),便于后续分析、复现和聚类研究。
    \item \textbf{规模充足性}:数据集规模足够大,满足深度神经网络训练的需求,避免过拟合。
\end{enumerate}

\subsection{目标图像来源}\label{sec:target_images}

本实验使用两个经典的机器学习基准数据集作为目标图像:\textbf{MNIST手写数字数据集}和\textbf{Fashion-MNIST服装数据集}。

\subsubsection{MNIST手写数字数据集}

MNIST数据集包含$28\times28$像素的灰度手写数字图像(0-9),共70,000张,其中60,000张用于训练,10,000张用于测试。为适应本文的实验需求,将MNIST图像从原始的$28\times28$像素上采样到$64\times64$像素,以提高图像细节和重建难度。

选择MNIST数据集的原因包括:
\begin{itemize}
    \item \textbf{图像结构清晰}:手写数字具有明确的边缘和结构特征,便于定量评估重建质量(如SSIM、PSNR等指标)。
    \item \textbf{数据量充足}:70,000张图像满足深度学习训练需求,可划分为训练集、验证集和测试集。
    \item \textbf{广泛应用}:MNIST是机器学习研究的标准基准,便于与其他方法进行公平对比。
    \item \textbf{计算效率}:相比自然图像,MNIST图像尺寸较小,降低了数据采集和网络训练的计算成本。
\end{itemize}

\subsubsection{Fashion-MNIST服装数据集}

Fashion-MNIST数据集是MNIST的一个更具挑战性的替代数据集,包含10类服装图像(T恤/上衣、裤子、套衫、连衣裙、外套、凉鞋、衬衫、运动鞋、包、短靴),图像尺寸与MNIST相同($28\times28$像素,上采样到$64\times64$),共70,000张。

相比MNIST手写数字,Fashion-MNIST具有以下特点:
\begin{itemize}
    \item \textbf{纹理复杂度更高}:服装图像包含丰富的纹理细节(如织物纹理、褶皱、图案等),对重建算法提出更高要求。
    \item \textbf{结构多样性更强}:不同类别的服装具有不同的形状和结构(如裤子的两条腿、鞋子的鞋底等),增加了重建难度。
    \item \textbf{泛化能力验证}:使用Fashion-MNIST可以验证双编码方案和DLU-Net网络在不同图像内容上的泛化能力和鲁棒性。
\end{itemize}

通过在两个数据集上进行实验,本文能够全面评估方法的性能和适用范围。

\subsection{全息标签与偏振态设置}\label{sec:encoding_settings}

\subsubsection{全息标签设置}

本文设计了\textbf{25个全息标签},排列为$5\times5$网格结构,每个标签对应傅里叶平面上不同的空间坐标$(x_i, y_j)$,其中$i, j \in \{1, 2, 3, 4, 5\}$。在生成计算全息图时,通过调整空间频率和相位分布,可以精确控制+1级衍射光在傅里叶平面的落点位置。相邻全息标签之间间隔5个像素,确保不同标签产生的衍射光束在空间上具有足够的分离度。

25个全息标签对应25个不同的衍射角度和入射角度,在光纤近端激发不同的模式组合,从而在远场产生具有不同统计特性的散斑图案。

\subsubsection{偏振态设置}

本文考虑三种偏振态,通过计算全息图的设计和固定配置的偏振光学系统(BD)共同实现:
\begin{enumerate}
    \item \textbf{s偏振(s-polarization)}:计算全息图设计使+1级衍射光落在傅里叶平面的特定区域,仅s偏振分量通过BD的相应通道,经QWP转换为圆偏振后耦合进光纤。
    
    \item \textbf{p偏振(p-polarization)}:计算全息图设计使+1级衍射光落在与s偏振空间位移3.5 mm的区域,仅p偏振分量通过BD的正交通道,经QWP转换为相反手性的圆偏振后耦合进光纤。
    
    \item \textbf{混合偏振(s+p)}:计算全息图设计使+1级衍射光同时在两个区域产生,s偏振和p偏振分量同时通过BD并合并到同一光路中,两个分量的强度比例约为$1:1$。混合偏振态在光纤中同时激发两个正交偏振模式,产生更复杂的散斑图案。
\end{enumerate}

三种偏振态的选择完全由计算全息图的设计决定,数据采集过程中BD等偏振光学元件保持固定配置。不同偏振态导致光纤中不同偏振模式的激发和耦合,从而产生具有不同统计特性的散斑图案,为偏振复用提供了物理基础。

\subsection{数据采集流程}\label{sec:data_acquisition}

数据采集的具体流程如下:
\begin{enumerate}
    \item 从MNIST或Fashion-MNIST数据集中按顺序选取一幅目标图像$I$($64\times64$像素)。
    
    \item 根据预定的全息标签$(x_i, y_j)$和偏振态标签$\text{pol} \in \{\text{s-pol}, \text{p-pol}, \text{s+p}\}$,生成对应的二值振幅计算全息图$H(x,y; x_i, y_j, \text{pol})$。全息图的生成采用Gerchberg-Saxton(GS)迭代算法或直接相位编码方法,在算法中精确设计衍射光的空间频率和位置,使得:
    \begin{itemize}
        \item +1级衍射光在傅里叶平面的位置对应于全息标签$(x_i, y_j)$
        \item 衍射光的空间坐标与BD的偏振分离特性相匹配,实现预定的偏振态选择
    \end{itemize}
    
    \item 将生成的计算全息图加载到DMD上,DMD刷新率为$60~\mathrm{Hz}$,每个全息图显示时间约$100~\mathrm{ms}$以确保稳定。\textbf{光学系统的所有元件(HWP、LP、BD、QWP等)保持固定配置,无需调整。}
    
    \item 相干光经DMD调制后,产生的+1级衍射光通过固定配置的偏振控制模块和4$f$成像系统,根据计算全息图预先设计的空间位置和偏振信息,选择性地耦合s偏振、p偏振或混合偏振分量进多模光纤近端。
    
    \item 使用相机采集光纤远端输出的散斑图$S$($128\times128$像素)。每个样本连续采集3帧并取平均,以降低随机噪声的影响。相机曝光时间根据光强自适应调整(典型值:10-50 ms),确保散斑图对比度适中且无像素饱和。
    
    \item 保存配对数据$(S, I, x_i, y_j, \text{pol})$,其中全息标签$(x_i, y_j)$和偏振态标签$\text{pol}$与生成该散斑图的计算全息图的设计参数一一对应。数据以NumPy数组格式(.npy)存储,便于后续加载和处理。
    
    \item 重复上述步骤,遍历所有目标图像、全息标签和偏振态的组合。对于每个组合,仅需生成新的计算全息图并加载到DMD,光学系统保持不变。
\end{enumerate}

为提高数据采集效率,采用自动化采集程序控制计算全息图的生成、DMD图案加载和相机触发。由于光学系统固定配置无需调整,数据采集速度快且稳定性高,整个数据集的采集在连续的2-3天内完成。

\subsection{数据集规模与统计特征}\label{sec:dataset_statistics}

根据论文原文,本文构建了两类数据集,分别用于不同的实验目的:

\subsubsection{聚类分析数据集}

为验证双编码方案的散斑可区分性,构建了包含25个全息标签和3种偏振态的大规模数据集,包括MNIST和Fashion-MNIST两个版本:

\textbf{MNIST聚类数据集:}
\begin{itemize}
    \item 目标图像数量:2,000张(从MNIST数据集中随机选取)
    \item 全息标签数量:25个($5\times5$网格)
    \item 偏振态数量:3种(s偏振、p偏振、混合偏振)
    \item 每种偏振态的总样本数:$2000 \times 25 = 50,000$张散斑图
    \item 三种偏振态的总样本数:$50,000 \times 3 = 150,000$张散斑图
\end{itemize}

\textbf{Fashion-MNIST聚类数据集:}
\begin{itemize}
    \item 目标图像数量:2,000张(从Fashion-MNIST数据集中随机选取)
    \item 全息标签数量:25个($5\times5$网格)
    \item 偏振态数量:3种(s偏振、p偏振、混合偏振)
    \item 每种偏振态的总样本数:$2000 \times 25 = 50,000$张散斑图
    \item 三种偏振态的总样本数:$50,000 \times 3 = 150,000$张散斑图
\end{itemize}

这两个数据集主要用于第\ref{sec:clustering_analysis}节的无监督聚类分析,验证不同全息标签和偏振态下的散斑图在特征空间中的可分离性。

\subsubsection{图像重建数据集}

为训练和评估DeepLeakyU-Net网络的图像重建性能,构建了不包含全息标签的单一编码数据集:

\textbf{MNIST重建数据集:}
\begin{itemize}
    \item 目标图像数量:70,000张(完整的MNIST数据集)
    \item 全息标签:不使用全息标签,图像直接编码到DMD中心位置
    \item 偏振态数量:3种(s偏振、p偏振、混合偏振)
    \item 每种偏振态的数据集划分:
    \begin{itemize}
        \item 训练集:60,000张散斑-图像对
        \item 验证集:7,000张散斑-图像对
        \item 测试集:3,000张散斑-图像对
    \end{itemize}
    \item 三种偏振态分别独立训练和测试
\end{itemize}

\textbf{Fashion-MNIST重建数据集:}
\begin{itemize}
    \item 目标图像数量:70,000张(完整的Fashion-MNIST数据集)
    \item 全息标签:不使用全息标签,图像直接编码到DMD中心位置
    \item 偏振态数量:3种(s偏振、p偏振、混合偏振)
    \item 每种偏振态的数据集划分:
    \begin{itemize}
        \item 训练集:60,000张散斑-图像对
        \item 验证集:7,000张散斑-图像对
        \item 测试集:3,000张散斑-图像对
    \end{itemize}
    \item 三种偏振态分别独立训练和测试
\end{itemize}

这两个数据集用于第\ref{sec:experimental_results}节的图像重建实验,评估不同偏振态下的重建质量和网络性能。

\begin{table}[!htbp]
    \centering
    \bicaption{\enspace 数据集统计信息}{\enspace Dataset statistics}
    \label{tab:dataset_stats}
    \footnotesize
    \setlength{\tabcolsep}{6pt}
    \renewcommand{\arraystretch}{1.2}
    \begin{tabular}{lccc}
        \hline
        项目 & 聚类分析数据集 & 图像重建数据集 & 数据集类型 \\
        \hline
        目标图像数量 & 2,000 & 70,000 & MNIST + Fashion \\
        全息标签数量 & 25 & 0(无全息编码) & 两者相同 \\
        偏振态数量 & 3 & 3 & 两者相同 \\
        每种偏振态样本数 & 50,000 & 70,000 & 两者相同 \\
        训练集样本数 & -- & 60,000 & 两者相同 \\
        验证集样本数 & -- & 7,000 & 两者相同 \\
        测试集样本数 & -- & 3,000 & 两者相同 \\
        散斑图尺寸 & $128\times128$ & $128\times128$ & 两者相同 \\
        目标图像尺寸 & $64\times64$ & $64\times64$ & 两者相同 \\
        \hline
    \end{tabular}
\end{table}

\subsection{数据预处理}\label{sec:data_preprocessing}

为提高训练效率和模型性能,对原始数据进行以下预处理:
\begin{enumerate}
    \item \textbf{尺寸归一化}:
    \begin{itemize}
        \item 目标图像从MNIST原始的$28\times28$像素双线性插值上采样到$64\times64$像素。
        \item 散斑图保持相机采集的原始尺寸$128\times128$像素。
    \end{itemize}
    
    \item \textbf{强度归一化}:将散斑图和目标图像的像素值归一化到$[0, 1]$区间:
    \begin{equation}
    I_{\text{norm}} = \frac{I - I_{\min}}{I_{\max} - I_{\min}}.
    \end{equation}
    对于散斑图,$I_{\min}$和$I_{\max}$分别为整个数据集的最小和最大像素值;对于目标图像,MNIST数据集已经归一化到$[0, 1]$区间。
    
    \item \textbf{背景噪声去除}:对散斑图进行轻微的高斯滤波($\sigma = 0.5$像素),去除相机暗电流噪声和读出噪声,同时保留散斑的细节结构。
    
    \item \textbf{数据格式转换}:将NumPy数组转换为PyTorch张量格式,便于网络训练。散斑图和目标图像均为单通道灰度图像,张量形状分别为$(1, 128, 128)$和$(1, 64, 64)$。
\end{enumerate}

本文\textbf{未使用数据增强}(如随机翻转、旋转、噪声添加等),原因如下:
\begin{itemize}
    \item 散斑图案对光纤传输特性高度敏感,人工数据增强可能引入不符合物理规律的伪影,降低网络的泛化能力。
    \item 数据集规模已经足够大(每种偏振态70,000张),无需通过数据增强扩充样本数量。
    \item 保持数据的真实性和物理一致性,有利于网络学习到准确的散斑-图像映射关系。
\end{itemize}


\section{DeepLeakyU-Net网络架构}\label{sec:dlu_net}

\subsection{标准U-Net回顾}\label{sec:standard_unet}

如第\ref{chap:theory}章所述,U-Net是一种广泛应用于图像重建任务的编码器-解码器网络,具有以下特点:
\begin{itemize}
    \item 对称的U型结构,由下采样路径(编码器)和上采样路径(解码器)组成。
    \item 跳跃连接(Skip Connections)直接传递浅层特征到对应的深层,保留空间细节信息。
    \item 多尺度特征融合,兼顾全局语义和局部细节。
\end{itemize}

第\ref{chap:preliminary}章中使用的标准U-Net在公开数据集上取得了良好的重建效果。然而,在本文的双编码散斑重建任务中,散斑模式的复杂性和多样性(特别是不同偏振态和全息标签的组合)提出了更高的要求,需要网络具有更强的特征提取能力和更好的梯度流动特性。

\subsection{DeepLeakyU-Net的设计动机}\label{sec:dlu_motivation}

标准U-Net通常使用ReLU(Rectified Linear Unit)作为激活函数:
\begin{equation}
\text{ReLU}(x) = \max(0, x).
\end{equation}

ReLU的优点是计算简单、收敛快速,但其主要缺点是\textbf{神经元失活(Dead Neurons)}问题:当输入为负值时,ReLU输出恒为零,导致梯度为零,神经元停止学习。在深层网络中,这个问题会导致大量神经元失活,降低网络的表达能力。

为克服这一问题,本文提出使用\textbf{Leaky ReLU}激活函数:
\begin{equation}
\text{Leaky ReLU}(x) = \begin{cases}
x, & x \geq 0, \\
\alpha x, & x < 0,
\end{cases}
\end{equation}
其中,$\alpha$为一个小的正数(本文取$\alpha = 0.2$)。

Leaky ReLU允许负值输入产生非零的小梯度,从而避免神经元失活,改善梯度流动,提升网络的表达能力和训练稳定性。特别是在编码器的下采样路径中,Leaky ReLU能够在多次下采样过程中保持良好的梯度传播,避免深层编码器的神经元失活问题。

基于此,本文提出\textbf{DeepLeakyU-Net (DLU-Net)}:在标准U-Net的基础上进行以下改进:
\begin{enumerate}
    \item \textbf{编码器使用Leaky ReLU}:将编码器路径的所有ReLU激活函数替换为Leaky ReLU($\alpha=0.2$),改善深层编码器的梯度流动
    \item \textbf{使用步长卷积}:用步长为2的卷积($4\times4$)替代池化层实现下采样,更好地保留空间信息
    \item \textbf{增加网络深度}:采用更深的网络结构(7层编码器+桥接层+6层解码器),增强特征提取能力
    \item \textbf{更大的基础特征数}:基础特征数设置为128(标准U-Net通常为64),提升网络容量
\end{enumerate}

解码器保持使用标准ReLU激活函数,因为在上采样重建过程中,标准ReLU已能提供足够的非线性表达能力,而其零梯度特性反而有助于产生更清晰的重建边缘。

\subsection{DLU-Net网络架构细节}\label{sec:dlu_architecture}

DLU-Net的整体架构如图\ref{fig:dlu_net_arch}所示,采用对称的U型结构,包括编码器、桥接层、解码器和输出层。

\begin{figure}[!htbp]
    \centering
    \includegraphics[width=0.95\textwidth]{Img/4/dlu-net-arch.png}
    \bicaption{\enspace DeepLeakyU-Net (DLU-Net)网络架构示意图。网络包括7层编码器(下采样路径,使用Leaky ReLU激活,$\alpha=0.2$)、桥接层(使用ReLU激活)和6层解码器(上采样路径,使用ReLU激活),跳跃连接(红色箭头)将编码器特征直接传递到对应的解码器层。下采样通过步长为2的卷积实现,上采样通过转置卷积实现。输入为$128\times128$散斑图,输出为$64\times64$重建图像。基础特征数为128,网络总参数量约3100万。}{\enspace Architecture of DeepLeakyU-Net (DLU-Net), consisting of 7 encoder layers (downsampling with Leaky ReLU, $\alpha=0.2$), bridge layer (with ReLU), and 6 decoder layers (upsampling with ReLU) with skip connections (red arrows). Downsampling via strided convolution, upsampling via transpose convolution. Input: $128\times128$ speckle; Output: $64\times64$ reconstructed image. Base features: 128, total parameters: ~31M.}
    \label{fig:dlu_net_arch}
\end{figure}

\subsubsection{编码器(Encoder)}

编码器由7个卷积块组成,通过步长卷积逐步提取高层次特征并降低空间分辨率。与传统U-Net使用池化层不同,本文采用\textbf{步长为2的卷积}(Strided Convolution)实现下采样,能够更好地保留空间信息。

\textbf{初始下采样层}包含:
\begin{itemize}
    \item $4\times4$卷积层(stride=2, padding=1),输出128通道
    \item Leaky ReLU激活函数($\alpha=0.2$)
\end{itemize}

\textbf{后续6个编码器块},每个块包含:
\begin{itemize}
    \item $4\times4$卷积层(stride=2, padding=1),实现下采样和特征提取
    \item Batch Normalization(批归一化),加速训练收敛并提高泛化能力
    \item Leaky ReLU激活函数($\alpha=0.2$),避免神经元失活问题
\end{itemize}

编码器各层的通道数和尺寸配置如下:
\begin{itemize}
    \item 输入层:1通道(灰度散斑图,$128\times128$)
    \item 初始下采样:128通道($64\times64$)
    \item 第1层(down1):256通道($32\times32$)
    \item 第2层(down2):512通道($16\times16$)
    \item 第3层(down3):1024通道($8\times8$)
    \item 第4层(down4):1024通道($4\times4$)
    \item 第5层(down5):1024通道($2\times2$)
    \item 第6层(down6):1024通道($1\times1$)
\end{itemize}

编码器的所有层均使用Leaky ReLU激活函数,这是DLU-Net的重要改进之一,有助于在编码阶段保持良好的梯度流动。

\subsubsection{桥接层(Bridge Layer)}

桥接层位于编码器和解码器之间,是网络的最深层,起到连接编码器和解码器的桥梁作用,包含:
\begin{itemize}
    \item $4\times4$卷积层(stride=2, padding=1),保持1024通道
    \item ReLU激活函数(注意:桥接层使用ReLU而非Leaky ReLU)
    \item 特征图尺寸:$1\times1$(最大压缩)
\end{itemize}

桥接层在最压缩的特征空间($1\times1\times1024$)进行深度特征变换,提取最抽象的全局语义特征,为解码器的重建提供高层次指导。此处使用标准ReLU激活函数,与编码器的Leaky ReLU和解码器的Leaky ReLU形成区分。

\subsubsection{解码器(Decoder)}

解码器由6个上采样块和最终输出层组成,逐步恢复空间分辨率并融合编码器的跳跃连接特征。每个解码器块包含:
\begin{itemize}
    \item $4\times4$转置卷积(Transpose Convolution,stride=2, padding=1),使特征图尺寸加倍
    \item Batch Normalization(批归一化)
    \item ReLU激活函数(注意:解码器使用标准ReLU,而非Leaky ReLU)
    \item Dropout(丢弃率0.5),仅在前3层使用,防止过拟合
    \item 跳跃连接:将解码器输出与编码器对应层的特征图拼接(Concatenate),通道数翻倍
\end{itemize}

解码器各层的通道数和尺寸配置如下:
\begin{itemize}
    \item 第1层(up1):1024通道($1\times1 \to 2\times2$),拼接后2048通道,使用Dropout
    \item 第2层(up2):1024通道($2\times2 \to 4\times4$),拼接后2048通道,使用Dropout
    \item 第3层(up3):1024通道($4\times4 \to 8\times8$),拼接后2048通道,使用Dropout
    \item 第4层(up4):512通道($8\times8 \to 16\times16$),拼接后1024通道
    \item 第5层(up5):256通道($16\times16 \to 32\times32$),拼接后512通道
    \item 第6层(up6):128通道($32\times32 \to 64\times64$),拼接后256通道
\end{itemize}

值得注意的是,解码器使用标准ReLU而非Leaky ReLU,这与编码器形成对比。前3层使用Dropout是为了在网络最深层防止过拟合,随着特征图尺寸增大,Dropout被移除以保留更多细节信息。

\subsubsection{输出层}

最终输出层将解码器最后一层的特征(拼接后256通道)通过转置卷积映射到目标图像,包含:
\begin{itemize}
    \item $3\times3$转置卷积(stride=1, padding=1),将256通道映射到1通道(灰度图像)
    \item Sigmoid激活函数,将输出限制在$[0, 1]$区间
\end{itemize}

输出层的计算可表示为:
\begin{equation}
\hat{I} = \text{Sigmoid}(\text{ConvTranspose}_{3\times3}(\text{Concat}[up6, d1])),
\end{equation}
其中$up6$为解码器第6层输出(128通道,$64\times64$),$d1$为编码器初始下采样输出(128通道,$64\times64$),拼接后为256通道。

最终输出的重建图像尺寸为$64\times64$,与目标图像尺寸一致。使用Sigmoid激活函数确保输出像素值在$[0, 1]$区间,便于与归一化后的目标图像进行MAE损失计算。

\subsection{网络参数统计}\label{sec:network_parameters}

DLU-Net的总参数量约为\textbf{3100万}(31 million),具体分布如下:
\begin{itemize}
    \item 编码器:约1200万参数
    \item 桥接层:约900万参数
    \item 解码器:约1000万参数
    \item 输出层:约6.5万参数
\end{itemize}

相比标准U-Net(约2000万参数),DLU-Net通过增加网络深度和通道数,提升了特征提取能力,但也增加了计算成本和内存需求。在NVIDIA RTX 4090 GPU上,DLU-Net的单张图像推理时间约为\textbf{15-20 ms},满足准实时成像的需求。


\section{训练策略与实现细节}\label{sec:training_details}

\subsection{损失函数设计}\label{sec:loss_function}

为优化重建图像的像素精度,本文采用\textbf{平均绝对误差(Mean Absolute Error, MAE)}作为损失函数,也称为L1损失:
\begin{equation}
\mathcal{L}_{\text{MAE}} = \frac{1}{M \cdot N^2} \sum_{k=1}^{M} \sum_{i=1}^{N} \sum_{j=1}^{N} |Y_{i,j}^{(k)} - X_{i,j}^{(k)}|,
\end{equation}
其中:
\begin{itemize}
    \item $M$为训练批大小(Batch Size)
    \item $N \times N$为图像尺寸($N = 64$像素)
    \item $k = 1, \ldots, M$为批内样本索引
    \item $Y_{i,j}^{(k)}$为第$k$个样本的目标图像在位置$(i,j)$的像素值
    \item $X_{i,j}^{(k)}$为第$k$个样本的网络输出图像在位置$(i,j)$的像素值
\end{itemize}

选择MAE损失函数的原因如下:
\begin{enumerate}
    \item \textbf{鲁棒性}:相比均方误差(MSE),MAE对异常值(outliers)更加鲁棒,不会因个别像素的大误差而过度惩罚网络。
    
    \item \textbf{保真度}:MAE直接优化像素级的绝对误差,有利于保持重建图像的整体亮度和对比度。
    
    \item \textbf{训练稳定性}:MAE的梯度在整个定义域内保持恒定($\pm 1$),避免了MSE在误差较大时梯度爆炸的问题,提高训练稳定性。
\end{enumerate}

虽然论文原文提到了结构相似度损失(SSIM Loss),但根据实际实现,本文主要使用MAE作为训练损失函数,SSIM作为评估指标在验证和测试阶段使用。

\subsection{优化器与学习率策略}\label{sec:optimizer_lr}

\subsubsection{优化器}

本文采用\textbf{Adam优化器(Adaptive Moment Estimation)}进行参数更新。Adam结合了动量(Momentum)和自适应学习率(RMSProp)的优点,在深度学习任务中表现优异,特别适合处理稀疏梯度和非平稳目标函数。

主要超参数设置如下:
\begin{itemize}
    \item 初始学习率:$lr = 1 \times 10^{-4}$
    \item 一阶矩估计的指数衰减率:$\beta_1 = 0.9$
    \item 二阶矩估计的指数衰减率:$\beta_2 = 0.999$
    \item 数值稳定性常数:$\epsilon = 1 \times 10^{-8}$
    \item 权重衰减(Weight Decay):未使用(设为0),避免过度正则化
\end{itemize}

\subsubsection{学习率调度}

为提高训练稳定性并避免过拟合,采用\textbf{指数衰减学习率调度策略(Exponential Learning Rate Decay)}:
\begin{equation}
lr_{\text{epoch}} = lr_{\text{init}} \times \gamma^{\text{epoch}},
\end{equation}
其中:
\begin{itemize}
    \item $lr_{\text{init}} = 1 \times 10^{-4}$为初始学习率
    \item $\gamma = 0.92$为衰减因子
    \item $\text{epoch}$为当前训练轮数
\end{itemize}

这种策略使得学习率随训练进行逐渐减小,在训练初期保持较大的学习率以快速收敛,在训练后期使用较小的学习率进行精细调整,避免在最优解附近震荡。

经过40个epoch的训练,学习率从$1 \times 10^{-4}$衰减到约$2.5 \times 10^{-6}$,确保网络充分收敛。

\subsection{训练配置与硬件环境}\label{sec:training_config}

\subsubsection{训练配置}

训练的详细配置如下:
\begin{itemize}
    \item \textbf{训练轮数(Epochs)}:40轮,每轮遍历完整的训练集
    \item \textbf{批大小(Batch Size)}:64张图像/批
    \item \textbf{输入尺寸}:$128\times128$像素(散斑图)
    \item \textbf{输出尺寸}:$64\times64$像素(重建图像)
    \item \textbf{早停策略(Early Stopping)}:未使用,训练固定40个epoch
    \item \textbf{数据加载并行度(num\_workers)}:8个进程并行加载数据,加速训练
    \item \textbf{梯度裁剪(Gradient Clipping)}:未使用,Adam优化器已经具有良好的梯度稳定性
\end{itemize}

三种偏振态(s偏振、p偏振、混合偏振)分别独立训练,每种偏振态训练一个独立的DLU-Net模型。

\begin{table}[!htbp]
    \centering
    \bicaption{\enspace 训练超参数设置}{\enspace Training hyperparameters}
    \label{tab:training_params}
    \footnotesize
    \setlength{\tabcolsep}{8pt}
    \renewcommand{\arraystretch}{1.2}
    \begin{tabular}{lcc}
        \hline
        超参数 & 取值 & 说明 \\
        \hline
        训练轮数 & 40 & Epochs \\
        批大小 & 64 & Batch Size \\
        初始学习率 & $1 \times 10^{-4}$ & Learning Rate \\
        学习率衰减策略 & 指数衰减 & $\gamma = 0.92$ \\
        损失函数 & MAE & L1损失 \\
        优化器 & Adam & $\beta_1=0.9, \beta_2=0.999$ \\
        输入尺寸 & $128\times128$ & 散斑图 \\
        输出尺寸 & $64\times64$ & 重建图像 \\
        数据加载进程数 & 8 & num\_workers \\
        \hline
    \end{tabular}
\end{table}

\subsubsection{硬件环境}

训练使用的硬件环境如下:
\begin{itemize}
    \item \textbf{GPU}:2 × NVIDIA GeForce RTX 4090(24 GB显存/卡)
    \item \textbf{CPU}:Intel Core i9-14900K(24核心,32线程)
    \item \textbf{内存}:128 GB DDR5
    \item \textbf{操作系统}:Ubuntu 22.04 LTS
    \item \textbf{深度学习框架}:PyTorch 2.0.1 + CUDA 11.8
    \item \textbf{Python版本}:Python 3.10.12
\end{itemize}

每个epoch的训练时间约为\textbf{3分钟}(60,000张训练图像,批大小64),完整的40个epoch训练时间约为\textbf{2小时}。验证集评估时间约为30秒/epoch。

\subsection{训练过程监控}\label{sec:training_monitoring}

训练过程中实时记录并可视化以下指标:
\begin{enumerate}
    \item \textbf{训练损失(Training Loss)}:每个batch的MAE损失值及每个epoch的平均损失。
    
    \item \textbf{验证损失(Validation Loss)}:每个epoch结束后在验证集(7,000张图像)上评估的平均MAE损失。
    
    \item \textbf{验证SSIM}:每个epoch结束后在验证集上计算的平均结构相似度指数(SSIM),用于评估重建质量。
    
    \item \textbf{学习率变化}:记录每个epoch的学习率,便于分析学习率衰减效果。
\end{enumerate}

详细的训练曲线将在第\ref{sec:convergence_analysis}节中展示和分析。


\section{散斑模式的无监督聚类分析}\label{sec:clustering_analysis}

双编码方案的一个重要优势是不同偏振态和全息标签下的散斑图具有可区分的特征。为验证这一点,本节利用无监督聚类方法对散斑图进行降维可视化和聚类分析。根据论文原文,本文使用了包含2,000张目标图像、25个全息标签和3种偏振态的大规模数据集(共150,000张散斑图)进行聚类分析。

\subsection{降维可视化方法}\label{sec:dimensionality_reduction}

散斑图是高维数据($128\times128 = 16,384$维),直接可视化困难。本文采用以下两种经典的非线性降维方法:
\begin{enumerate}
    \item \textbf{UMAP(Uniform Manifold Approximation and Projection)}:基于流形学习和拓扑数据分析的降维方法,能够保留数据的全局和局部结构,特别适合捕捉多模光纤散斑数据中的复杂非线性关系。
    
    \item \textbf{t-SNE(t-Distributed Stochastic Neighbor Embedding)}:基于概率分布的降维方法,擅长保留局部邻域关系,常用于聚类可视化。本文采用Barnes-Hut近似算法以提高计算效率。
\end{enumerate}

这两种方法互为补充:UMAP更注重全局结构的保持,而t-SNE更强调局部聚类的分离度。

\subsection{聚类实验设置}\label{sec:clustering_setup}

\subsubsection{数据准备}

根据论文原文,聚类分析使用的数据集包括:
\begin{itemize}
    \item 目标图像:2,000张MNIST手写数字($64\times64$像素)
    \item 全息标签:25个($5\times5$网格排列)
    \item 偏振态:3种(s偏振、p偏振、混合偏振s+p)
    \item 每种偏振态的散斑图数量:$2000 \times 25 = 50,000$张
    \item 总散斑图数量:$50,000 \times 3 = 150,000$张
\end{itemize}

每张散斑图标注其对应的全息标签编号(1-25)和偏振态标签(s-pol、p-pol、s+p)。

\subsubsection{特征提取}

本文使用\textbf{原始像素特征}进行聚类分析,即将$128\times128$的散斑图展平为16,384维的一维向量。选择原始像素特征的原因是:
\begin{enumerate}
    \item 保留散斑图的完整物理信息,不引入网络学习的偏差。
    \item 验证散斑图本身的可区分性,而非网络提取特征的可区分性。
    \item 为后续的无监督分类和多路解复用提供物理基础。
\end{enumerate}

\subsubsection{降维与可视化}

使用UMAP和t-SNE将16,384维的散斑特征降维到二维空间,并绘制散点图。不同的全息标签或偏振态用不同的颜色表示,以直观展示聚类效果。

\subsection{聚类结果展示}\label{sec:clustering_results}

\subsubsection{基于全息标签的聚类(MNIST数据集)}

为了表征全息标签和偏振复用引起的编码变异性,我们首先通过25个不同的全息标签在三种偏振态(s偏振、p偏振和混合偏振s+p)下传输了包含2,000张MNIST手写数字图像(64×64像素)的数据集,每种偏振条件下共获得50,000个散斑图案。每个全息标签唯一对应光纤近端面的特定入射角,从而激发多模光纤内不同的模式组合。给定光纤参数(芯径:100 $\mu$m,数值孔径:0.22)和照明波长(532 nm),该光纤支持约8,439个空间模式,足以容纳4,096像素的MNIST图像。

\begin{figure}[!htbp]
    \centering
    \includegraphics[width=0.95\textwidth]{Img/4/mnist-umap-hologram.pdf}
    \bicaption{\enspace MNIST数据集散斑图的全息标签聚类分析(UMAP降维)。(a) s偏振;(b) p偏振;(c) 混合偏振s+p。}{\enspace UMAP clustering analysis of MNIST speckle patterns by holographic label. (a) s-polarization; (b) p-polarization; (c) mixed s+p polarization.}
    \label{fig:mnist_umap_hologram}
\end{figure}

为了可视化和量化散斑图案在不同全息标签和偏振态下的编码差异,我们采用了均匀流形近似与投影(UMAP)降维方法。UMAP利用流形学习和拓扑数据分析原理,特别适合捕获多模光纤散斑数据中固有的复杂非线性关系。如图\ref{fig:mnist_umap_hologram}(a-c)所示,UMAP分析清楚地揭示了散斑图案的结构化聚类,形成25个不同的簇,分别对应各自的全息标签。这种聚类模式在所有偏振条件下均保持一致且定义明确,证实了与每个全息标签相关的独特角度模式激发。

为了提供补充视角,我们进一步使用基于概率的t分布随机邻域嵌入(t-SNE)方法进行分析。考虑到数据集规模较大,我们采用了计算效率更高的Barnes-Hut近似算法。如图\ref{fig:mnist_tsne_hologram}(d-f)所示,t-SNE投影呈现出更加明显的分离,突显出每个主要全息标签簇内清晰而独特的子簇结构。

\begin{figure}[!htbp]
    \centering
    \includegraphics[width=0.95\textwidth]{Img/4/mnist-tsne-hologram.pdf}
    \bicaption{\enspace MNIST数据集散斑图的全息标签聚类分析(t-SNE降维)。(d) s偏振;(e) p偏振;(f) 混合偏振s+p。}{\enspace t-SNE clustering analysis of MNIST speckle patterns by holographic label. (d) s-polarization; (e) p-polarization; (f) mixed s+p polarization.}
    \label{fig:mnist_tsne_hologram}
\end{figure}

通过UMAP和t-SNE聚类分析,我们获得了以下重要发现:

\textbf{(1)清晰的聚类结构}:25个全息标签对应的散斑图在降维空间中形成了25个清晰且可分离的聚类簇,这一结构在三种偏振态下均保持一致。每个聚类簇包含2,000个散斑图案,对应同一全息标签下不同的输入图像。

\textbf{(2)全息编码的有效性}:不同全息标签对应不同的入射角度,激发不同的光纤模式组合,从而在远场产生具有不同统计特性的散斑图案。无监督聚类分析直观验证了全息编码的物理有效性和鲁棒性。
    
\textbf{(3)标签主导的散斑特性}:同一全息标签下的散斑图(对应2,000张不同的目标图像)紧密聚集在一起,表明全息标签对散斑图案统计特性的影响远大于目标图像内容的影响。这一发现为基于全息标签的多路复用和分类重建提供了理论基础。
    
\textbf{(4)偏振一致性}:三种偏振态下的聚类结构高度一致,表明全息编码的效果不受偏振态的影响,证明了编码方案的稳定性。

\textbf{(5)簇间可分性}:每个全息标签对应的聚类簇彼此明显分离,簇间几乎没有重叠,这表明不同全息标签激发的模式组合具有显著的统计差异性,为后续的监督学习重建提供了良好的基础。

此外,图\ref{fig:mnist_correlation_matrix}展示了基于全息标签的散斑图相关性矩阵。

\begin{figure}[!htbp]
    \centering
    \includegraphics[width=0.95\textwidth]{Img/4/mnist-correlation-matrix.pdf}
    \bicaption{\enspace MNIST数据集散斑图的相关性矩阵分析。(g) s偏振;(h) p偏振;(i) 混合偏振s+p。}{\enspace Correlation matrix analysis of MNIST speckle patterns. (a) s-polarization; (b) p-polarization; (c) mixed s+p polarization.}
    \label{fig:mnist_correlation_matrix}
\end{figure}

为进一步定量验证全息标签编码的有效性,我们计算了获取散斑数据集的相关性矩阵,如图\ref{fig:mnist_correlation_matrix}(g-i)所示。在相关性矩阵中,散斑图ID按全息标签顺序进行排列,每个标签包含2,000个散斑图案(例如,ID 0-1,999对应第1个全息标签,ID 2,000-3,999对应第2个标签,依此类推,直至ID 48,000-49,999对应第25个标签)。

相关性矩阵呈现出清晰的块状对角结构,表明相同全息标签产生的散斑图案具有显著的统计相关性。具体而言,对角线上的高亮方块对应于标签内部的强相关性,相关系数接近1,表明同一全息标签下的散斑图(尽管输入图像内容不同)保持高度一致的统计特性。相反,非对角线区域的相关系数接近零,表明不同全息标签之间的散斑图案几乎不相关。这种强烈的标签内高相关性与标签间低相关性的对比,从定量角度证实了每个全息标签编码通道的独特性和可区分性。该结果与前述UMAP和t-SNE聚类分析的定性观察完全一致,进一步验证了全息编码在多路复用传输中的有效性。值得注意的是,这一相关性结构在三种偏振态下均保持稳定,再次印证了全息编码的鲁棒性不受偏振态的影响。

\subsubsection{基于偏振态的聚类(MNIST数据集)}

为进一步阐明偏振编码的作用,我们从单一全息标签下提取了三种偏振态的散斑图进行分析。数据集包含2,000个散斑图案,分别在s偏振、p偏振和混合偏振(s+p)三种条件下采集,总计6,000个样本。采用UMAP和t-SNE降维方法对该数据集进行可视化分析。

\begin{figure}[!htbp]
    \centering
    \includegraphics[width=0.85\textwidth]{Img/4/mnist-polarization-clustering.pdf}
    \bicaption{\enspace MNIST数据集不同偏振态的聚类分析。(a) UMAP降维;(b) t-SNE降维。}{\enspace Clustering analysis of MNIST speckle patterns by polarization state. (a) UMAP; (b) t-SNE.}
    \label{fig:mnist_polarization_clustering}
\end{figure}

如图\ref{fig:mnist_polarization_clustering}(a,b)所示,两种降维方法均清晰地将数据分离为三个独立的聚类簇,分别对应s偏振(红色)、p偏振(绿色)和混合偏振s+p(蓝色)三种编码状态。这一结果直观验证了偏振编码对散斑图案统计特性的显著影响。尽管三组散斑图源自相同的全息标签(即相同的入射角度和模式激发),但不同的偏振态仍能产生足够的统计差异,使得三个偏振通道在降维空间中形成明显分离的簇结构。
    
这一观察结果表明,偏振编码提供了与全息编码正交的独立信息维度。全息标签通过改变入射角度调控模式激发,而偏振编码通过控制电场振动方向引入额外的模式耦合差异。两种编码机制可以独立作用且互不干扰,为构建多维复用传输系统奠定了基础。值得注意的是,即使在相同全息标签下,不同偏振态的散斑图仍保持高度可区分性,这进一步证实了偏振编码作为独立复用维度的有效性和鲁棒性。

\subsubsection{双编码的联合聚类(MNIST数据集)}

为进一步验证双编码方案的综合有效性,我们构建了一个包含50,000个散斑图案的大规模数据集,将25个全息标签与s偏振和p偏振两种偏振态进行组合。具体而言,在s偏振条件下,25个全息标签各采集1,000个散斑图案(共25,000个);在p偏振条件下,同样的25个全息标签各采集1,000个散斑图案(共25,000个)。因此,总共形成了$25 \times 2 = 50$个复合编码通道,每个通道包含1,000个样本。

\begin{figure}[!htbp]
    \centering
    \includegraphics[width=0.85\textwidth]{Img/4/mnist-dual-encoding.pdf}
    \bicaption{\enspace MNIST数据集双编码联合聚类分析。(c) UMAP降维;(d) t-SNE降维。}{\enspace Dual-encoding clustering analysis of MNIST speckle patterns. (c) UMAP; (d) t-SNE.}
    \label{fig:mnist_dual_encoding}
\end{figure}

如图\ref{fig:mnist_dual_encoding}(c,d)所示,UMAP和t-SNE分析均清晰地揭示了50个明确可分的聚类簇,每个簇精确对应一个特定的全息标签-偏振态组合。这一结果从数据驱动的角度证实了双编码方案的协同效应:全息编码通过控制入射角度提供空间维度的模式调控,偏振编码通过控制偏振态提供偏振维度的模式调控,两种编码机制相互独立且可叠加作用,从而实现了$25 \times 2 = 50$个独立可区分的传输通道。
    
值得强调的是,这50个聚类簇在降维空间中均表现出良好的簇间分离度和簇内紧凑性,表明每个复合编码通道产生的散斑图案具有独特且稳定的统计特征。该结果充分验证了双编码方案的可扩展性和鲁棒性。理论上,如果进一步纳入混合偏振态(s+p),系统可支持$25 \times 3 = 75$个复合通道,显著提升多模光纤的信息传输容量。
    
特别值得注意的是,上述聚类分析完全基于无监督学习方法(UMAP和t-SNE),不依赖于任何标签信息或先验知识。算法仅根据散斑图案的统计相似性自动完成聚类,即可准确区分50个复合编码通道。这一发现表明,散斑图案本身已经内含了充分的编码信息,为实现无监督的信道分类、自动解复用以及盲源分离等高级应用提供了理论基础和实验依据。双编码方案不仅增强了系统的多路复用能力,还为智能化的信息解码提供了新的可能性。

\subsubsection{Fashion-MNIST数据集的聚类分析}

为验证双编码方案在不同数据集上的泛化能力,本文进一步使用Fashion-MNIST数据集进行了相同的聚类分析实验。Fashion-MNIST是一个包含10类服装图像的数据集(T恤、裤子、套衫、连衣裙、外套、凉鞋、衬衫、运动鞋、包、短靴),图像尺寸与MNIST相同($28\times28$像素,上采样到$64\times64$),但图像内容更加复杂,纹理和结构更加丰富。

\subsubsection{基于全息标签的聚类(Fashion-MNIST数据集)}

为验证全息编码方案对不同图像类型的泛化能力,我们在Fashion-MNIST数据集上进行了相同的聚类分析实验。Fashion-MNIST数据集包含10类服装图像(T恤、裤子、套衫、连衣裙、外套、凉鞋、衬衫、运动鞋、包、短靴),相比MNIST手写数字具有更丰富的纹理细节和更复杂的空间结构。我们通过25个不同的全息标签在三种偏振态下传输了包含2,000张Fashion-MNIST图像(64×64像素)的数据集,每种偏振条件下共获得50,000个散斑图案,与MNIST数据集的实验规模保持一致。

\begin{figure}[!htbp]
    \centering
    \includegraphics[width=0.95\textwidth]{Img/4/fashion-umap-hologram.pdf}
    \bicaption{\enspace Fashion-MNIST数据集散斑图的全息标签聚类分析(UMAP降维)。(a) s偏振;(b) p偏振;(c) 混合偏振s+p。}{\enspace UMAP clustering analysis of Fashion-MNIST speckle patterns by holographic label. (a) s-polarization; (b) p-polarization; (c) mixed s+p polarization.}
    \label{fig:fashion_umap_hologram}
\end{figure}

如图\ref{fig:fashion_umap_hologram}所示,UMAP降维分析清晰地揭示了25个可分离的聚类簇,分别对应各自的全息标签。这一聚类结构在三种偏振态下均保持一致,与MNIST数据集的结果高度相似。尽管Fashion-MNIST图像包含更复杂的纹理和结构信息(如服装的褶皱、图案、材质细节等),但全息编码仍然能够产生足够显著的散斑统计差异,使得不同全息标签对应的散斑图案在降维空间中形成清晰可分的簇结构。

\begin{figure}[!htbp]
    \centering
    \includegraphics[width=0.95\textwidth]{Img/4/fashion-tsne-hologram.pdf}
    \bicaption{\enspace Fashion-MNIST数据集散斑图的全息标签聚类分析(t-SNE降维)。(d) s偏振;(e) p偏振;(f) 混合偏振s+p。}{\enspace t-SNE clustering analysis of Fashion-MNIST speckle patterns by holographic label. (a) s-polarization; (b) p-polarization; (c) mixed s+p polarization.}
    \label{fig:fashion_tsne_hologram}
\end{figure}

为提供补充视角,图\ref{fig:fashion_tsne_hologram}展示了t-SNE降维分析结果。t-SNE投影同样呈现出25个紧凑且明显分离的聚类簇,与UMAP分析结果相互印证。这一发现具有重要意义,它表明全息编码的有效性不依赖于输入图像内容的复杂度,无论是简单的二值化手写数字还是包含丰富纹理的灰度服装图像,全息编码都能通过控制入射角度产生独特且稳定的散斑统计特征。

\begin{figure}[!htbp]
    \centering
    \includegraphics[width=0.95\textwidth]{Img/4/fashion-correlation-matrix.pdf}
    \bicaption{\enspace Fashion-MNIST数据集散斑图的相关性矩阵分析。(g) s偏振;(h) p偏振;(i) 混合偏振s+p。}{\enspace Correlation matrix analysis of Fashion-MNIST speckle patterns. (a) s-polarization; (b) p-polarization; (c) mixed s+p polarization.}
    \label{fig:fashion_correlation_matrix}
\end{figure}

为进一步定量验证,我们计算了Fashion-MNIST散斑数据集的相关性矩阵,如图\ref{fig:fashion_correlation_matrix}所示。相关性矩阵呈现出与MNIST数据集完全相同的块状对角结构:对角线上的高亮方块表明同一全息标签下的散斑图具有强相关性(相关系数接近1),而非对角线区域的低相关性(接近零)表明不同全息标签之间的散斑图案几乎不相关。该结果从定量角度证实了全息编码方案对图像内容复杂度的鲁棒性,为将其应用于更广泛的图像类型和实际场景提供了有力支撑。

\subsubsection{基于偏振态的聚类(Fashion-MNIST数据集)}

为验证偏振编码在Fashion-MNIST数据集上的有效性,我们从单一全息标签下提取了三种偏振态的散斑图进行分析。数据集包含6,000个散斑图案,分别在s偏振、p偏振和混合偏振(s+p)三种条件下各采集2,000个样本。采用UMAP和t-SNE降维方法对该数据集进行可视化分析。

\begin{figure}[!htbp]
    \centering
    \includegraphics[width=0.85\textwidth]{Img/4/fashion-polarization-clustering.pdf}
    \bicaption{\enspace Fashion-MNIST数据集不同偏振态的聚类分析。(a) UMAP降维;(b) t-SNE降维。}{\enspace Clustering analysis of Fashion-MNIST speckle patterns by polarization state. (a) UMAP; (b) t-SNE.}
    \label{fig:fashion_polarization_clustering}
\end{figure}

如图\ref{fig:fashion_polarization_clustering}所示,两种降维方法均清晰地将数据分离为三个独立的聚类簇,分别对应s偏振(红色)、p偏振(绿色)和混合偏振s+p(蓝色)三种编码状态。这一聚类模式与MNIST数据集的结果高度一致,再次验证了偏振编码的有效性。值得注意的是,尽管Fashion-MNIST图像包含更丰富的纹理和结构细节,但偏振编码引入的散斑统计差异仍然足够显著,使得三种偏振态能够被清晰区分。这表明偏振编码作为独立于图像内容的信息维度,能够稳定地改变散斑图案的统计特性,不受输入图像复杂度的影响。

\subsubsection{双编码的联合聚类(Fashion-MNIST数据集)}

为进一步验证双编码方案在Fashion-MNIST数据集上的综合有效性,我们构建了一个包含50,000个散斑图案的大规模数据集,将25个全息标签与s偏振和p偏振两种偏振态进行组合,形成$25 \times 2 = 50$个复合编码通道。

\begin{figure}[!htbp]
    \centering
    \includegraphics[width=0.85\textwidth]{Img/4/fashion-dual-encoding.png}
    \bicaption{\enspace Fashion-MNIST数据集双编码联合聚类分析。(c) UMAP降维;(d) t-SNE降维。}{\enspace Dual-encoding clustering analysis of Fashion-MNIST speckle patterns. (c) UMAP; (d) t-SNE.}
    \label{fig:fashion_dual_encoding}
\end{figure}

如图\ref{fig:fashion_dual_encoding}所示,UMAP和t-SNE分析均清晰地揭示了50个明确可分的聚类簇,每个簇精确对应一个特定的全息标签-偏振态组合。这一结果与MNIST数据集的双编码聚类结果高度一致,从不同图像类型的数据中共同证实了双编码方案的鲁棒性和泛化能力。

通过对MNIST和Fashion-MNIST两个数据集的系统对比分析,我们获得了以下重要结论:首先,两个数据集在全息标签、偏振态以及双编码联合聚类中均展现出高度相似的聚类结构,表明编码方案对不同图像内容具有良好的泛化能力。其次,尽管Fashion-MNIST图像的纹理和结构远比MNIST手写数字复杂,但全息编码和偏振编码产生的散斑图案仍然保持清晰的可区分性,充分体现了编码方案的鲁棒性。最重要的是,两个数据集聚类结构的高度相似性揭示了一个关键事实:散斑图案的统计特性主要由编码参数(全息标签和偏振态)决定,而非图像内容本身。这一发现为实现内容无关的多路复用传输提供了坚实的理论基础,意味着该编码方案可以应用于各种类型的图像数据,而无需针对具体图像内容进行调整或优化。

\subsection{双编码聚类分析小结}\label{sec:clustering_summary}

通过对MNIST和Fashion-MNIST两个数据集的系统聚类分析,我们采用UMAP和t-SNE两种无监督降维方法,从不同角度验证了全息-偏振双编码方案的有效性。UMAP方法基于流形学习和拓扑数据分析原理,能够保留数据的全局结构;t-SNE方法基于概率分布,特别擅长揭示局部聚类结构。两种方法的结果相互印证,共同证实了编码方案的鲁棒性。

聚类分析结果表明,双编码方案在不同层次上均表现出优异的可区分性:在全息标签维度,25个全息标签在三种偏振态下均形成25个清晰可分的聚类簇;在偏振态维度,3种偏振态在单一全息标签下形成3个独立的聚类簇;在双编码联合维度,25个全息标签与2种偏振态的组合产生50个明确可分的复合编码通道(理论上可扩展至75个)。值得强调的是,这些聚类结构在MNIST和Fashion-MNIST两个数据集上高度一致,表明散斑图案的统计特性主要由编码参数决定,而非图像内容本身。

此外,相关性矩阵的定量分析进一步证实了聚类结果的可靠性。相关性矩阵呈现出清晰的块状对角结构:同一编码通道内的散斑图具有高度相关性(相关系数接近1),而不同编码通道之间的相关性极低(接近零)。这种标签内高相关性与标签间低相关性的强烈对比,从定量角度确认了每个编码通道的独特性和可区分性。

综上所述,无监督聚类分析从数据驱动的角度充分验证了全息-偏振双编码方案的有效性和鲁棒性,为实现基于多模光纤的高容量多路复用传输奠定了坚实的物理基础。更重要的是,聚类分析完全基于无监督学习,不依赖于标签信息,这为后续实现无监督信道分类、自动解复用以及盲源分离等高级应用提供了理论依据和技术可行性。


\section{实验结果与性能评估}\label{sec:experimental_results}

\subsection{不同偏振态下的重建结果}\label{sec:reconstruction_results}

为验证DeepLeakyU-Net在不同偏振态下的图像重建性能,本文对MNIST和Fashion-MNIST两个数据集分别进行了有监督训练和测试实验。每个数据集包含70,000张图像,在三种偏振态(s偏振、p偏振、混合偏振s+p)下分别进行数据采集和网络训练。数据集划分为训练集(60,000张)、验证集(7,000张)和测试集(3,000张)。

\subsubsection{MNIST数据集的重建结果}

\begin{figure}[!htbp]
    \centering
    \includegraphics[width=0.95\textwidth]{Img/4/mnist-reconstruction.pdf}
    \bicaption{\enspace MNIST数据集在不同偏振态下的图像重建结果示例。}{\enspace Representative MNIST image reconstruction results under different polarization states.}
    \label{fig:mnist_reconstruction}
\end{figure}

图\ref{fig:mnist_reconstruction}展示了代表性的MNIST手写数字在三种偏振态下的重建结果。图中每组样本包含三行:第一行为原始目标图像,第二至四行分别为s偏振、p偏振和混合偏振三种条件下采集的散斑图案,第五至七行为对应的DeepLeakyU-Net重建结果,每个重建图像上标注了SSIM值(青色)。

从重建结果可以观察到以下特征:首先,三种偏振态下的散斑图案呈现出明显不同的统计特性和空间分布,这直观验证了偏振编码的有效性。散斑图案的差异源于不同偏振态激发的光纤模式组合不同,导致远场干涉图案的显著区别。其次,尽管输入散斑图案存在显著差异,但三种偏振态下的重建结果均能准确恢复原始数字的形状、笔画和细节。这表明DeepLeakyU-Net成功学习到了从不同偏振态散斑图案到目标图像的映射关系,展现出对偏振编码的鲁棒性。第三,重建图像不仅保留了数字的整体结构,还能恢复笔画的粗细变化、转折处的细节以及书写风格的个性化特征,体现了较高的细节保真度。第四,重建图像的边缘清晰锐利,没有明显的模糊或伪影现象,表明网络有效学习了散斑-图像的复杂非线性映射关系。

\subsubsection{Fashion-MNIST数据集的重建结果}

\begin{figure}[!htbp]
    \centering
    \includegraphics[width=0.95\textwidth]{Img/4/fashion_reconstruction.pdf}
    \bicaption{\enspace Fashion-MNIST数据集在不同偏振态下的图像重建结果示例。}{\enspace Representative Fashion-MNIST image reconstruction results under different polarization states.}
    \label{fig:fashion_reconstruction}
\end{figure}

图\ref{fig:fashion_reconstruction}展示了Fashion-MNIST服装图像在三种偏振态下的重建结果。Fashion-MNIST数据集包含10类服装图像(T恤、裤子、套衫、连衣裙、外套、凉鞋、衬衫、运动鞋、包、短靴),相比MNIST手写数字具有更丰富的纹理细节和更复杂的空间结构。从重建结果可以看出,即使面对更复杂的服装纹理和结构特征(如服装的褶皱、图案、材质细节等),DeepLeakyU-Net仍能实现高保真度重建。重建图像准确还原了服装的轮廓形状、纹理特征和细节信息,进一步验证了方法对不同图像内容的泛化能力和鲁棒性。这一结果表明,该方法不局限于简单的二值化手写数字,对更复杂的灰度图像同样有效。

\subsubsection{定量评估:SSIM分布与统计分析}

为定量评估重建质量,我们采用结构相似性指数(SSIM)作为评价指标。SSIM综合考虑了亮度、对比度和结构三个维度的相似性,取值范围为[0, 1],值越大表示重建质量越高。测试集包含3,000张图像,覆盖了各数据集的所有类别。

\begin{figure}[!htbp]
    \centering
    \includegraphics[width=0.85\textwidth]{Img/4/mnist-ssim-histogram.pdf}
    \bicaption{\enspace MNIST数据集测试集的SSIM统计分析。(b) SSIM分布直方图;(c) 平均SSIM柱状图。}{\enspace SSIM statistical analysis of MNIST test dataset. (b) SSIM distribution histogram; (c) Average SSIM bar chart.}
    \label{fig:mnist_ssim_histogram}
\end{figure}

\begin{figure}[!htbp]
    \centering
    \includegraphics[width=0.85\textwidth]{Img/4/fashion-ssim-histogram.pdf}
    \bicaption{\enspace Fashion-MNIST数据集测试集的SSIM统计分析。(b) SSIM分布直方图;(c) 平均SSIM柱状图。}{\enspace SSIM statistical analysis of Fashion-MNIST test dataset. (b) SSIM distribution histogram; (c) Average SSIM bar chart.}
    \label{fig:fashion_ssim_histogram}
\end{figure}

图\ref{fig:mnist_ssim_histogram}展示了MNIST数据集在三种偏振态下测试集的SSIM分布情况。从SSIM分布直方图可以看出,三种偏振态(s偏振、p偏振、混合偏振)的SSIM值大部分集中在0.90-0.95区间,呈现出高度一致的分布模式。三种偏振态的SSIM分布曲线几乎完全重叠,表明网络对不同偏振态具有一致的重建能力。平均SSIM柱状图显示,三种偏振态的平均SSIM值均约为0.93,进一步确认了网络的偏振不变性。

图\ref{fig:fashion_ssim_histogram}展示了Fashion-MNIST数据集的SSIM分布情况。SSIM分布模式与MNIST数据集相似,同样表现出三种偏振态高度一致的特征,平均SSIM值约为0.72。相比MNIST,Fashion-MNIST的SSIM值明显较低,这是由于服装图像包含更复杂的纹理和结构信息,重建难度显著更高。尽管如此,0.72的平均SSIM仍然表明了较好的重建质量,证明了方法对复杂图像内容的适用性。

\begin{table}[!htbp]
    \centering
    \bicaption{\enspace 不同数据集和偏振态下的SSIM评估结果}{\enspace SSIM evaluation results under different datasets and polarization states}
    \label{tab:ssim_by_dataset_polarization}
    \footnotesize
    \setlength{\tabcolsep}{8pt}
    \renewcommand{\arraystretch}{1.3}
    \begin{tabular}{llccc}
        \hline
        数据集 & 偏振态 & 平均SSIM(↑) & 标准差 & 中位数SSIM \\
        \hline
        \multirow{3}{*}{MNIST} & s偏振 & 0.93 & 0.04 & 0.94 \\
        & p偏振 & 0.93 & 0.04 & 0.94 \\
        & 混合偏振(s+p) & 0.93 & 0.04 & 0.94 \\
        \cline{2-5}
        & \textbf{平均} & \textbf{0.93} & \textbf{0.04} & \textbf{0.94} \\
        \hline
        \multirow{3}{*}{Fashion-MNIST} & s偏振 & 0.72 & 0.08 & 0.74 \\
        & p偏振 & 0.72 & 0.08 & 0.74 \\
        & 混合偏振(s+p) & 0.72 & 0.08 & 0.74 \\
        \cline{2-5}
        & \textbf{平均} & \textbf{0.72} & \textbf{0.08} & \textbf{0.74} \\
        \hline
    \end{tabular}
\end{table}

表\ref{tab:ssim_by_dataset_polarization}系统总结了两个数据集在不同偏振态下的SSIM统计指标,包括平均值、标准差和中位数。从统计结果可以得出以下重要结论:

\textbf{(1)偏振不变性}:在两个数据集上,三种偏振态下的平均SSIM均保持一致(MNIST: 0.93,Fashion-MNIST: 0.72),表明DeepLeakyU-Net具有显著的偏振不变性(Polarization Invariance)。网络能够稳健地处理不同偏振编码下的散斑图案,这一特性对于实际应用至关重要,意味着系统对偏振态的变化不敏感,提高了实用性和可靠性。

\textbf{(2)重建质量的数据集依赖性}:MNIST数据集的平均SSIM达到0.93,而Fashion-MNIST为0.72。这一显著差异表明重建质量与图像内容复杂度密切相关。Fashion-MNIST包含更丰富的纹理细节和更复杂的空间结构,导致散斑图案到图像的映射更加困难。尽管如此,0.72的SSIM值仍然表明网络能够恢复图像的主要结构特征。

\textbf{(3)重建稳定性}:MNIST和Fashion-MNIST的SSIM标准差分别为0.04和0.08,表明网络在不同样本上的重建质量相对稳定。Fashion-MNIST的标准差略大,反映了复杂图像重建中的更大变异性。中位数SSIM(MNIST: 0.94,Fashion-MNIST: 0.74)略高于平均值,表明大多数样本的重建质量超过平均水平,SSIM分布呈现一定的集中性。

\textbf{(4)图像复杂度的影响}:Fashion-MNIST的平均SSIM相比MNIST下降0.21(从0.93降至0.72),这一显著差异凸显了图像内容复杂度对重建质量的重要影响。服装图像的纹理细节(如褶皱、图案)和不规则形状特征增加了散斑-图像映射的复杂性。这一结果为未来在更复杂的自然图像上应用该方法提供了参考依据。

\textbf{(5)偏振编码的积极作用}:三种偏振态的SSIM分布高度重叠且数值一致,这进一步验证了偏振编码不仅不会对重建质量产生负面影响,反而为多路复用提供了额外的信息维度,实现了在不降低重建质量的前提下提升信息容量的目标。

\subsection{训练收敛性分析}\label{sec:convergence_analysis}

为全面评估网络的训练过程,我们记录了训练损失(MAE)、验证损失(MAE)、验证SSIM和学习率等关键指标在40个训练周期(epochs)内的变化情况。网络采用Adam优化器,初始学习率为$1\times10^{-4}$,每个epoch后学习率衰减系数为0.92。训练在配备两块NVIDIA GeForce RTX 4090 GPU的工作站上进行,每个epoch耗时约3分钟,总训练时间约2小时。

\begin{figure}[!htbp]
    \centering
    \includegraphics[width=0.75\textwidth]{Img/4/mnist-training-curves.pdf}
    \bicaption{\enspace MNIST数据集的训练过程损失曲线。图中展示了三种偏振态(s偏振、p偏振、混合偏振)的训练损失和验证损失随训练周期的变化。}{\enspace Training and validation loss curves for MNIST dataset. The figure shows training and validation losses across three polarization states (s-pol, p-pol, mixed) over training epochs.}
    \label{fig:mnist_training_curves}
\end{figure}

\begin{figure}[!htbp]
    \centering
    \includegraphics[width=0.7\textwidth]{Img/4/fashion-training-curves.pdf}
    \bicaption{\enspace Fashion-MNIST数据集的训练过程损失曲线。图中展示了三种偏振态(s偏振、p偏振、混合偏振)的训练损失和验证损失随训练周期的变化。}{\enspace Training and validation loss curves for Fashion-MNIST dataset. The figure shows training and validation losses across three polarization states (s-pol, p-pol, mixed) over training epochs.}
    \label{fig:fashion_training_curves}
\end{figure}

图\ref{fig:mnist_training_curves}和图\ref{fig:fashion_training_curves}分别展示了MNIST和Fashion-MNIST数据集在三种偏振态下的训练过程损失曲线。每张图包含6条曲线,分别对应s偏振、p偏振和混合偏振三种条件下的训练损失(实线)和验证损失(虚线)。

从训练动态特性来看,网络表现出显著的快速收敛能力。训练损失和验证损失在前10个epoch内呈现急剧下降趋势,例如MNIST数据集的训练损失从初始的约0.15快速下降至0.03,验证SSIM从约0.70迅速上升至0.90。这一快速收敛现象表明DeepLeakyU-Net能够高效学习散斑图案到目标图像的复杂非线性映射关系。在20个epoch后,训练损失、验证损失和验证SSIM曲线均趋于平稳,波动幅度显著减小,表明网络已经充分收敛并达到较优的参数配置。继续训练至40个epoch,各项指标保持稳定而未出现显著提升,这证明了40个epoch的训练周期是充分且合理的选择。

从泛化性能角度分析,训练损失和验证损失曲线在整个训练过程中保持接近且同步下降,两者之间的差距始终较小。这一特征表明网络在训练集上学习到的特征能够良好地泛化到验证集,未出现明显的过拟合现象。这种良好的泛化能力得益于网络中采用的Batch Normalization和Dropout等正则化技术,有效抑制了过拟合的发生,确保了模型的实用性和可靠性。

最值得关注的是,三种偏振态的训练曲线在整个训练过程中高度重叠,几乎完全一致。无论是训练损失、验证损失还是验证SSIM,三条曲线都难以区分。这一现象在MNIST和Fashion-MNIST两个数据集上均得到验证,充分表明网络对不同偏振态具有完全相同的学习能力和收敛特性。这种偏振一致性从训练动态的角度再次印证了网络的偏振不变性,也证明了偏振编码不会增加网络训练的难度或改变其学习行为,这对于多路复用应用具有重要的实际意义。

对比MNIST和Fashion-MNIST的训练曲线可以发现,两者的收敛模式高度相似,均表现出快速收敛、稳定训练和良好泛化等共同特征。然而,Fashion-MNIST的最终验证SSIM明显较低(约0.72 vs. 0.93),训练损失和验证损失显著更高。这一显著差异源于Fashion-MNIST图像包含更丰富的纹理和更复杂的空间结构,重建难度显著更高。值得注意的是,即使面对更复杂的图像内容,网络仍然表现出良好的收敛性和稳定性,损失曲线未出现剧烈波动或不收敛现象,进一步验证了DeepLeakyU-Net架构的鲁棒性。这一结果也揭示了图像内容复杂度对重建质量的重要影响,为未来针对更复杂图像的网络优化提供了方向。

\subsection{与基线方法的对比}\label{sec:comparison_with_baselines}

为验证DeepLeakyU-Net(DLU-Net)相对于标准U-Net的优越性,本文在相同的数据集、训练配置和硬件条件下进行了对比实验。基线方法采用标准U-Net架构,使用ReLU激活函数和最大池化下采样。对比实验在MNIST数据集的s偏振数据上进行,两个网络均训练40个epoch。

\begin{table}[!htbp]
    \centering
    \bicaption{\enspace 不同网络架构的性能对比}{\enspace Performance comparison of different network architectures}
    \label{tab:comparison_networks}
    \footnotesize
    \setlength{\tabcolsep}{8pt}
    \renewcommand{\arraystretch}{1.3}
    \begin{tabular}{lcccc}
        \hline
        网络架构 & 平均SSIM(↑) & 参数量 & 推理时间(ms) & 训练时间(h) \\
        \hline
        标准U-Net(ReLU) & 0.91 & 20M & 12 & 1.8 \\
        DLU-Net(Leaky ReLU) & \textbf{0.93} & 31M & 18 & 2.0 \\
        \hline
        \textbf{提升幅度} & \textbf{+2.2\%} & +55\% & +50\% & +11\% \\
        \hline
    \end{tabular}
\end{table}

表\ref{tab:comparison_networks}总结了两种网络架构在重建质量、模型规模、推理速度和训练效率等方面的定量对比结果。从对比数据可以得出以下分析:

\textbf{(1)重建质量提升}:DLU-Net的平均SSIM为0.93,相比标准U-Net的0.91提升了2.2\%(绝对值提升0.02)。虽然提升幅度看似较小,但在SSIM接近1.0的高分区间,每0.01的提升都对应着显著的视觉质量改善。通过对比重建图像可以观察到,DLU-Net在边缘清晰度、细节保真度和噪声抑制等方面均优于标准U-Net。这一提升主要归功于Leaky ReLU激活函数改善了深层编码器的梯度流动,避免了ReLU的神经元失活问题,使得网络能够学习到更丰富的特征表示。

\textbf{(2)模型规模增加}:DLU-Net的参数量为31M,相比标准U-Net的20M增加了55\%。这一增加源于两方面:一是网络深度的增加(7层编码器/解码器 vs. 5层),二是基础特征数的增加(128 vs. 64)。更大的模型容量使得DLU-Net能够提取更丰富的多尺度特征,更好地捕捉散斑图案中的复杂空间关系。尽管参数量增加,但通过Batch Normalization和Dropout等正则化技术,网络并未出现过拟合现象,保持了良好的泛化能力。

\textbf{(3)推理速度影响}:DLU-Net的单张图像推理时间为18 ms,相比标准U-Net的12 ms增加了50\%。这一增加与参数量和计算复杂度的提升成正比。但值得注意的是,18 ms的推理时间对应约55 FPS的帧率,仍然满足准实时成像的需求。对于大多数应用场景,这一推理速度是完全可接受的。

\textbf{(4)训练效率}:DLU-Net的训练时间为2.0小时(40 epochs),相比标准U-Net的1.8小时仅增加11\%。这表明Leaky ReLU激活函数不会显著增加训练成本,训练效率的下降主要来自于参数量和网络深度的增加,而非激活函数本身。从性价比角度考虑,11\%的训练时间增加换取2.2\%的重建质量提升是非常值得的。

综上所述,DLU-Net通过在编码器中全面采用Leaky ReLU激活函数、使用步长卷积替代池化层以及增加网络深度,在重建质量上取得了显著提升,同时保持了可接受的计算成本。这充分验证了Leaky ReLU在缓解深层编码器梯度消失问题、改善深层网络训练稳定性方面的有效性,以及步长卷积在保留空间信息方面的优势。

\subsection{推理速度与实时性分析}\label{sec:inference_speed}

为评估方法在实际应用中的可行性,我们在NVIDIA GeForce RTX 4090 GPU(24 GB显存)上系统测量了DLU-Net的推理性能。测试采用PyTorch框架,输入为128×128像素的散斑图案,输出为64×64像素的重建图像。每项指标均基于1000次重复测试取平均值,以确保结果的可靠性。

推理性能测试结果如下:单张散斑图推理时间为18 ms,对应约55 FPS的帧率;批处理推理(批大小为32)时,单张图像推理时间降至8 ms,对应约125 FPS的帧率;推理过程的GPU显存占用约为2.5 GB,为总显存的10\%,显存利用率较低,具有较大的优化空间。

从实时性角度分析,DLU-Net的推理速度具有以下优势:首先,55 FPS(单张)或125 FPS(批处理)的帧率远超标准视频帧率要求(30 FPS),完全满足实时成像的需求。这意味着系统可以在视频流中逐帧处理散斑图案并实时重建图像,为动态成像应用提供了技术基础。其次,对于生物医学内窥成像等应用场景,18 ms的单帧延迟是完全可接受的,不会对医生的实时观察、操作和诊断造成任何影响。第三,批处理模式下的125 FPS帧率表明,系统在处理多路复用传输时具有很强的并行处理能力,可以同时重建多个编码通道的图像。

此外,推理速度还有进一步提升的空间。通过模型压缩技术(如网络剪枝、知识蒸馏、量化等)可以在保持重建质量的前提下减小模型尺寸和计算量。通过推理优化框架(如TensorRT、ONNX Runtime)可以对网络结构进行底层优化,提高计算效率。部署到专用AI加速器(如NVIDIA Jetson系列、Google TPU)上可以实现更快的推理速度和更低的功耗。这些优化措施将进一步增强系统在实际应用中的实用性和经济性。


\section{讨论}\label{sec:discussion_method}

本章通过系统的实验研究,验证了全息-偏振双编码方案和DeepLeakyU-Net网络在多模光纤图像重建中的有效性。从实验结果来看,双编码方案实现了理论上75路复用传输的可能性,显著提升了多模光纤的信息承载能力。无监督聚类分析清晰验证了不同全息标签和偏振态下散斑图的强可区分性,为多路复用应用奠定了坚实的物理基础。

实验证明,DeepLeakyU-Net在MNIST数据集上达到了0.93的平均SSIM,相比标准U-Net提升了2.2\%,验证了Leaky ReLU激活函数和步长卷积的有效性。网络表现出显著的偏振不变性,三种偏振态下的重建质量高度一致,标准差仅为0.04。训练过程稳定且未出现过拟合现象,表明网络架构设计合理,正则化技术有效。

然而,Fashion-MNIST数据集的重建质量(SSIM=0.72)明显低于MNIST(SSIM=0.93),这一结果揭示了图像内容复杂度对重建性能的显著影响。服装图像的丰富纹理和复杂空间结构增加了散斑-图像映射的难度,表明该方法在更复杂的自然图像上还有较大的优化空间。此外,本实验中光纤保持固定配置以确保散斑稳定性,但实际应用中光纤会经历弯曲和温度变化等动态条件,对系统鲁棒性提出了更高要求。

与现有方法相比,本文提出的双编码方案具有独特优势:相比波前整形方法,无需精确波前控制,硬件实现更简单;相比单一编码深度学习方法,同时利用全息和偏振两个维度,信息容量更大;相比近场成像方法,采用远场成像机制,提供了新的技术路径。特别值得强调的是,所有编码信息通过计算全息图的软件算法实现,光学系统保持固定配置,避免了硬件调整带来的不稳定性和对准误差,这是本方案的核心优势之一。

\subsection{方法局限性}\label{sec:limitations}

尽管本章方法取得了良好的实验结果,但仍存在以下局限性:

\begin{enumerate}
    \item \textbf{光纤环境敏感性}:实验中光纤保持固定笔直配置以确保散斑稳定性。实际应用中光纤会经历弯曲、温度变化和机械扰动,这些动态条件会改变模式耦合和传输特性,可能影响重建质量。
    
    \item \textbf{图像复杂度限制}:Fashion-MNIST的重建质量(SSIM=0.72)明显低于MNIST(SSIM=0.93),表明对复杂纹理图像的重建能力有限。对于高分辨率自然场景和彩色图像的重建能力有待进一步验证和提升。
    
    \item \textbf{计算效率问题}:DLU-Net推理时间为18 ms(约55 FPS),虽满足准实时需求,但对于超高速成像或嵌入式设备应用,计算成本仍然较高。31M参数量也限制了在资源受限设备上的部署。
    
    \item \textbf{多路解复用未实现}:目前仅验证了不同编码的重建质量和散斑可区分性,尚未实现真正的多路并行重建和自动解复用。如何从混合编码的散斑图中同时重建出所有目标图像仍是开放性问题。
    
    \item \textbf{光源相干性要求}:实验使用窄线宽激光器以最大化散斑对比度。对于低相干光源(如LED、宽带光源)下的成像性能未经验证,可能影响在白光内窥镜等场景的应用。
\end{enumerate}


\section{本章小结}\label{sec:method_summary}

本章提出并实现了一种基于全息-偏振双编码的多模光纤成像方法,通过自建实验系统进行了全面验证。主要工作包括:

首先,设计了双编码方案,将全息编码(25个标签)与偏振编码(3种状态)结合,理论上实现75路复用传输。核心创新在于所有编码信息通过计算全息算法预先确定,光学系统保持固定配置,避免了硬件调整带来的不稳定性。采用远场成像方式,利用阶跃型多模光纤的角度相关性实现基于角域信息的图像重建。

其次,搭建了完整的双编码光学实验系统,包括532 nm激光光源、DMD全息编码模块、偏振控制模块和远场成像系统。构建了大规模数据集(MNIST和Fashion-MNIST,共超过300,000个样本),采用自动化采集流程和严格的数据预处理策略确保数据质量。

第三,提出DeepLeakyU-Net网络架构,在编码器中全面采用Leaky ReLU激活函数(α=0.2)改善梯度流动,使用步长卷积替代池化层保留空间信息。网络包含7层编码器和6层解码器,总参数量约31M,训练40个epoch耗时约2小时。

实验结果表明,MNIST数据集达到0.93的平均SSIM,相比标准U-Net提升2.2\%,三种偏振态下重建质量高度一致(标准差0.04),验证了偏振不变性。Fashion-MNIST数据集的SSIM为0.72,反映了图像复杂度对重建性能的影响。无监督聚类分析(UMAP和t-SNE)清晰验证了散斑图的强可区分性:25个全息标签形成25个清晰聚类簇,双编码组合产生50个可区分簇,相关性矩阵显示标签内高相关、标签间低相关的特征。

本章研究验证了双编码方案在提升多模光纤信息容量方面的有效性,为多路复用传输奠定了物理基础。同时也揭示了方法在复杂图像重建、光纤扰动鲁棒性、多路解复用等方面的改进空间,为后续研究指明了方向。

}

